% -*- Mode: LaTeX; Package: CLIM-USER -*-

\chapter {Incremental Redisplay}
\label {incremental-redisplay}

\section {Overview of Incremental Redisplay}

CLIM's incremental redisplay facility to allows the programmer to change the
output in an output history (and hence, on the screen or other output device) in
an incremental fashion.  It allows the programmer to redisplay individual pieces
of the existing output differently, under program control.  It is
``incremental'' in the sense that CLIM will try to minimize the changes to the
existing output on a display device when displaying new output.

There are two different ways to do incremental redisplay.

The first is to call \cl{redisplay} on an output record.  In essence, this tells
CLIM to recompute the output of that output record over from scratch.  CLIM
compares the new results with the existing output and tries to do minimal
redisplay.  The \cl{updating-output} form allows the programmer to assist CLIM
by informing it that entire branches of the output history are known not to have
changed.  \cl{updating-output} also allows the programmer to communicate the
fact that a piece of the output record hierarchy has moved, either by having an
output record change its parent, or by having an output record change its
position.

The second way to do incremental redisplay is for the programmer to manually do the
updates to the output history, and then call \cl{note-output-record-child-changed}
on an output record.  This causes CLIM to propagate the changes up the output
record tree and allows parent output records to re-adjust themselves to account for
the changes.

Each style is appropriate under different circumstances.  \cl{redisplay} is
often easier to use, especially when there might be large numbers of changes
between two passes, or when the programmer has only a poor idea as to what the
changes might be.  \cl{note-output-record-child-changed} can be more efficient
for small changes at the bottom of the output record hierarchy, or in cases
where the programmer is well informed as to the specific changes necessary and
can help CLIM out.


\subsection {Examples of Incremental Redisplay}

The usual technique of incremental redisplay is to use \cl{updating-output} to
inform CLIM what output has changed, and use \cl{redisplay} to recompute and
redisplay that output.

The outermost call to \cl{updating-output} identifies a program fragment that
produces incrementally redisplayable output.  A nested call to
\cl{updating-output} (that is, a call to \cl{updating-output} that occurs during
the execution of the body of the outermost \cl{updating-output} and specifies
the same stream) identifies an individually redisplayable piece of output, the
program fragment that produces that output, and the circumstances under which
that output needs to be redrawn.  This nested calls to \cl{updating-output} are
just hints to incremental redisplay that can reduce the amount of work done by
CLIM.

The outermost call to \cl{updating-output} executes its body, producing the
initial version of the output, and returns an \cl{updating-output-record} that
captures the body in a closure.  Each nested call to \cl{updating-output} stores
its \cl{:unique-id} and \cl{:cache-value} arguments and the portion of the
output produced by its body.

\cl{redisplay} takes an \cl{updating-output-record} and executes the captured
body of \cl{updating-output} over again.  When a nested call to
\cl{updating-output} is executed during redisplay, \cl{updating-output} decides
whether the cached output can be reused or the output needs to be redrawn.  This
is controlled by the \cl{:cache-value} argument to \cl{updating-output}.  If its
value matches its previous value, the body would produce output identical to the
previous output and thus it is unnecessary for CLIM to execute the body again.
In this case the cached output is reused and \cl{updating-output} does not
execute its body.  If the cache value does not match, the output needs to be
recomputed, so \cl{updating-output} executes its body and the new output drawn
on the stream replaces the previous output.  The \cl{:cache-value} argument is
only meaningful for nested calls to \cl{updating-output}.

In order to compare the cache to the output record, two pieces of information
are necessary:

\begin{itemize}
\item An association between the output being done by the program and a
particular cache.  This is supplied in the \cl{:unique-id} option to
\cl{updating-output}.

\item A means of determining whether this particular cache is valid.  This is
the \cl{:cache-value} option to \cl{updating-output}.
\end{itemize}

Normally, the programmer would supply both options. The unique-id would be some
data structure associated with the corresponding part of output.  The cache
value would be something in that data structure that changes whenever the output
changes.

It is valid to give the \cl{:unique-id} and not the \cl{:cache-value}.  This is
done to identify a parent in the hierarchy.  By this means, the children
essentially get a more complex unique id when they are matched for output.  (In
other words, it is like using a telephone area code.)  The cache without a cache
value is never valid.  Its children always have to be checked.

It is also valid to give the \cl{:cache-value} and not the \cl{:unique-id}.  In
this case, unique ids are just assigned sequentially.  So, if output associated
with the same thing is done in the same order each time, it isn't necessary to
invent new unique ids for each piece.  This is especially true in the case of
children of a cache with a unique id and no cache value of its own.  In this
case, the parent marks the particular data structure, whose components can
change individually, and the children are always in the same order and properly
identified by their parent and the order in which they are output.

A unique id need not be unique across the entire redisplay, only among the
children of a given output cache; that is, among all possible (current and
additional) uses made of \cl{updating-output} that are dynamically (not
lexically) within another.

To make incremental redisplay maximally efficient, the programmer should attempt
to give as many caches with \cl{:cache-value} as possible.  For instance, if the
thing being redisplayed is a deeply nested tree, it is better to be able to know
when whole branches have not changed than to have to recurse to every single
leaf and check it.  So, if there is a modification tick in the leaves, it is
better to also have one in their parent of the leaves and propagate the
modification up when things change.  While the simpler approach works, it
requires CLIM to do more work than is necessary.

The following function illustrates the standard use of incremental redisplay:

\begin{verbatim}
(defun test (stream)
  (let* ((list (list 1 2 3 4 5))
         (record
           (updating-output (stream)
             (do* ((elements list (cdr elements))
                   (count 0 (1+ count)))
                  ((null elements))
               (let ((element (first elements)))
                 (updating-output (stream :unique-id count
                                          :cache-value element)
                   (format stream "Element ~D" element)
                   (terpri stream)))))))
    (sleep 10)
    (setf (nth 2 list) 17)
    (redisplay record stream)))
\end{verbatim}

When this function is run on a window, the initial display will look like:

\begin{verbatim}
  Element 1
  Element 2
  Element 3
  Element 4
  Element 5
\end{verbatim}

After the sleep has terminated, the display will look like:

\begin{verbatim}
  Element 1
  Element 2
  Element 17
  Element 4
  Element 5
\end{verbatim}

CLIM takes care of ensuring that only the third line gets erased and
redisplayed.  In the case where items moved around (try the example substituting

\begin{verbatim}
(setq list (sort list #'(lambda (x y)
                          (declare (ignore x y))
                          (zerop (random 2))))) 
\end{verbatim}

for the form after the call to \cl{sleep}), CLIM would ensure that the minimum
amount of work would be done in updating the display, thereby minimizing
``flashiness'' while providing a powerful user interface.

See Chapter~\ref{application-frames} for a discussion of how to use incremental
redisplay automatically within the panes of an application frame.


\section {Standard Programmer Interface}

\Defmacro {updating-output} {(stream
                              \rest args
                              \key unique-id (id-test \#'\cl{eql})
                                   cache-value (cache-test \#'\cl{eql})
                                   fixed-position all-new parent-cache
                                   record-type)
                             \body body}

Introduces a caching point for incremental redisplay.  

The \arg{stream} argument is not evaluated, and must be a symbol that is bound to
an output recording stream.  If \arg{stream} is \cl{t}, \cl{*standard-output*} is
used.  \arg{body} may have zero or more declarations as its first forms.

\arg{record-type} specifies the class of output record to create.  The default
is \cl{standard-updating-output-record}.  This argument should only be supplied
by a programmer if there is a new class of output record that supports the
updating output record protocol.

\cl{updating-output} must be implemented by expanding into a call to
\cl{invoke-updating-output}, supplying a function that executes \arg{body} as
the \arg{continuation} argument to \cl{invoke-updating-output}.  The exact
behavior of this macro is described under \cl{invoke-updating-output}.

\Defgeneric {invoke-updating-output} {stream continuation record-type
                                      unique-id id-test
                                      cache-value cache-test
                                      \key all-new parent-cache}

Introduces a caching point for incremental redisplay.  Calls the function
\arg{continuation}, which generates the output records to be redisplayed.
\arg{continuation} is a function of one argument, the stream; it has dynamic
extent.

If this is used outside the dynamic scope of an incremental redisplay, it has no
particular effect.  However, when incremental redisplay is occurring, the
supplied \arg{cache-value} is compared with the value stored in the cache
identified by \arg{unique-id}.  If the values differ or the code in \arg{body}
has not been run before, the code in \arg{body} runs, and \arg{cache-value} is
saved for next time.  If the cache values are the same, the code in \arg{body}
is not run, because the current output is still valid.

\arg{unique-id} provides a means to uniquely identify the output done by
\arg{body}.  If \arg{unique-id} is not supplied, CLIM will generate one that is
guaranteed to be unique.  \arg{unique-id} may be any object as long as it is
unique with respect to the \arg{id-test} predicate among all such unique ids in
the current incremental redisplay.  \arg{id-test} is a function of two arguments
that is used for comparing unique ids; it has indefinite extent.

\arg{cache-value} is a value that remains constant if and only if the output
produced by body does not need to be recomputed.  If the cache value is not
supplied, CLIM will not use a cache for this piece of output.  \arg{cache-test}
is a function of two arguments that is used for comparing cache values; it has
indefinite extent.

If \arg{fixed-position} is \term{true}, then the location of this output is
fixed relative to its parent output record.  When CLIM redisplays an output
record that has a fixed position, then if the contents have not changed, the
position of the output record will not change.  If the contents have changed,
CLIM assumes that the code will take care to preserve its position.  The default
for \arg{fixed-position} is \term{false}.

If \arg{all-new} is \term{true}, that indicates that all of the output done by
\arg{body} is new, and will never match output previously recorded.  In this
case, CLIM will discard the old output and do the redisplay from scratch.  The
default for \arg{all-new} is \term{false}.

The output record tree created by \cl{updating-output} defines a caching
structure where mappings from a unique-id to an output record are maintained.
If the programmer specifies an output record $P$ via the
\arg{parent-cache} argument, then CLIM will try to find a corresponding output
record with the matching unique-id in the cache belonging to $P$.  If 
\arg{parent-cache} is not provided, then CLIM looks for the unique-id in the
output record created by immediate dynamically enclosing call to
\cl{updating-output}.  If that fails, CLIM uses the unique-id to find an output
record that is a child of the output history of \arg{stream}.  Once CLIM has
found an output record that matches the unique-id, it uses the cache value and
cache test to determine whether the output record has changed.  If the output
record has not changed, it may have moved, in which case CLIM will simply move
the display of the output record on the display device.


\Defun {redisplay} {record stream \key (check-overlapping \cl{t})}

This function simply calls \cl{redisplay-output-record} on the arguments
\arg{record} and \arg{stream}.

\Defgeneric {redisplay-output-record} {record stream
                                       \optional (check-overlapping \cl{t}) 
                                                 x y parent-x parent-y}  

\issue {SWM} {The coordinate system stuff affected by the x/y and parent-x/y
arguments is entirely bogus.  The proposal to make ``stream relative''
coordinates for output records instead of ``parent relative'' coordinates will
eliminate this completely.}

\cl{(redisplay-output-record \arg{record} \arg{stream})} causes the output of
\arg{record} to be recomputed.
CLIM redisplays the changes ``incrementally'', that is, it only displays those
parts that have been changed. \arg{record} must already be part of the output
history of the \term{output recording stream} \arg{stream}, although it can be
anywhere inside the hierarchy.

When \arg{check-overlapping} is \term{false}, this means that CLIM can assume
that no sibling output records overlap each other at any level in the output
record tree.  Supplying a \term{false} value for this argument can improve
performance of redisplay.

{\bf Implementation note:} \cl{redisplay-output-record} is implemented by first
binding \cl{stream-redisplaying-p} of the stream to \term{true}, then creating
the new output records by invoking \cl{compute-new-output-records}.  Once the
new output records have been computed, \cl{compute-difference-set} is called to
compute the difference set, which is then passed to
\cl{note-child-output-record-changed}.

The other optional arguments can be used to specify where on the \arg{stream}
the output record should be redisplayed.  \arg{x} and \arg{y} represent where
the cursor should be, relative to (\cl{output-record-parent} record), before we
start redisplaying \arg{record}.  \arg{parent-x} and \arg{parent-y} can be
supplied to say: do the output as if the parent started at positions
\arg{parent-x} and \arg{parent-y} (which are in absolute coordinates).  The
default values for \arg{x} and \arg{y} are \cl{(output-record-start-position
\arg{record})}.  The default values for \arg{parent-x} and \arg{parent-y} are

\begin{verbatim}
(convert-from-relative-to-absolute-coordinates 
    stream
    (output-record-parent record))
\end{verbatim}

\arg{record} will usually be an output record created by \cl{updating-output}.
If it is not, then \cl{redisplay-output-record} will be equivalent to
\cl{replay-output-record}.


\section {Incremental Redisplay Protocol}

\Issue {SWM} {While the description of the API here is accurate, the description
of the protocol is a disaster.  This is no surprise, since the protocol for
increment redisplay is itself a disaster.}

\Defprotoclass {updating-output-record}

The protocol class corresponding to records that support incremental redisplay;
a subclass of \cl{output-record}.
\IfYouWantClass {an} {updating output record} {updating-output-record}

\Defpredicate {updating-output-record-p} {object}

Returns \term{true} if \arg{object} is an \term{updating output record},
otherwise returns \term{false}.

\definitarg {:unique-id}
\definitarg {:id-test}
\definitarg {:cache-value}
\definitarg {:cache-test}
\Definitarg {:fixed-position}

All subclasses of \cl{updating-output-record} must handle these four initargs,
which are used to specify, respectively, the unique id and id test, cache value
and cache test, and the ``fixed position'' component of the output record.

\Defclass {standard-updating-output-record}

The instantiable class of output record that supports incremental redisplay.
This is a subclass of \cl{updating-output-record}.


\Defgeneric {output-record-unique-id} {record}

Returns the unique id associated with the updating output record \arg{record}.

\Defgeneric {output-record-cache-value} {record}

Returns the cache value associated with the updating output record \arg{record}.

\Defgeneric {output-record-fixed-position} {record}

Returns \term{true} if the updating output record \arg{record} is at a fixed
location on the output stream, otherwise returns \term{false}.  Output records
that are not at fixed location on the output stream will be moved by incremental
redisplay when any of their siblings adjust their size or position.

\Defgeneric {output-record-displayer} {record}

Returns the function that produces the output for this output record.  This is
the function that is called during redisplay to produce new output if the cache
value mismatches.


\Defgeneric {compute-new-output-records} {record stream}

\cl{compute-new-output-records} modifies an output record tree to reflect new
output done by the application.  In addition to inserting the new output records
into the output record tree, it must save enough information to be able to
compute the difference set, such as the old bounding rectangle, old cursor
positions, old children, and so forth.

\cl{compute-new-output-records} recursively invokes itself on each child of
\arg{record}.

\cl{compute-new-output-records} of an output record of type
\cl{updating-output-record} runs the displayer (\cl{output-record-displayer}),
which gives the behavior of incremental redisplay.  That is, it reruns the code
(getting hints from \cl{updating-output}) and figures out the changes from there
by comparing it to the old output history.


\Defgeneric {compute-difference-set} {record \optional (check-overlapping \cl{t})
                                      (offset-x \cl{0}) (offset-y \cl{0})
                                      (old-offset-x \cl{0}) (old-offset-y \cl{0})} 

\cl{compute-difference-set} compares the current state of the \term{output
record} \arg{record} with its previous state, and returns a ``difference set''
as five values.  The difference set controls what needs to be done to the
display device in order to accomplish the incremental redisplay.

The values returned are \arg{erases} (what areas of the display device need to
be erased), \arg{moves} (what output records need to be moved), \arg{draws}
(what output records need to be freshly replayed), \arg{erase-overlapping}, and
\arg{move-overlapping}.  Each is a list whose elements are lists of the form:

\begin{itemize}
\item \arg{erases} are lists of \cl{(\arg{record} \arg{old-box})}

\item \arg{moves} are lists of \cl{(\arg{record} \arg{old-box} \arg{new-position})}

\item \arg{draws} are lists of \cl{(\arg{record} \arg{old-box})}

\item \arg{erase-overlapping} is a list of \cl{(\arg{record} \arg{old-box})}

\item \arg{move-overlapping} is a list of \cl{(\arg{record} \arg{old-box} \arg{new-position})}
\end{itemize}

When \arg{check-overlapping} is \term{false}, this means that CLIM can assume
that no sibling output records overlap each other at any level.  Supplying a
\term{false} value for this argument can improve performance of redisplay.


\Defgeneric {augment-draw-set} {record erases moves draws erase-overlapping move-overlapping
                                \optional x-offset y-offset old-x-offset old-y-offset}

\issue {SWM} {To be supplied.}


\Defgeneric {note-output-record-child-changed}
            {record child mode old-position old-bounding-rectangle stream
             \optional erases moves draws erase-overlapping move-overlapping
             \key check-overlapping}

\cl{note-output-record-child-changed} is called after an output history has had
changes made to it, but before any of the new output has been displayed.  It
will call \cl{propagate-output-record-changes-p} to determine if the parent
output record should be notified, and if so, will call
\cl{propagate-output-record-changes} to create an updated difference set.  If no
changes need to be propagated to the parent output record, then
\cl{note-output-record-child-changed} will call \cl{incremental-redisplay} in
order to display the difference set.

\arg{mode} is one of \cl{:delete}, \cl{:add}, \cl{:change}, \cl{:move}, or
\cl{:none}.

\arg{old-position} and \arg{old-bounding-rectangle} describe where \arg{child}
was before it was moved.

\arg{check-overlapping} is as for \cl{compute-difference-set}.


\Defgeneric {propagate-output-record-changes-p} {record child mode 
                                                 old-position old-bounding-rectangle}

\cl{propagate-output-record-changes-p} is a predicate that returns \term{true}
if the change made to the child will cause \arg{record} to be redisplayed in any
way.  Otherwise, it returns \term{false}.  \arg{mode} is one of \cl{:delete},
\cl{:add}, \cl{:change}, \cl{:move}, or \cl{:none}.

\Defgeneric {propagate-output-record-changes} 
            {record child mode 
             \optional old-position old-bounding-rectangle
                       erases moves draws erase-overlapping move-overlapping check-overlapping}

Called when the changed \arg{child} output record requires that its parent,
\arg{record}, be redisplayed as well.  \cl{propagate-output-record-changes}
will update the difference set to reflect the additional changes.

\arg{check-overlapping} is as for \cl{compute-difference-set}.


\Defgeneric {match-output-records} {record \rest initargs}

Returns \term{true} if record matches the supplied class initargs
\arg{initargs}, otherwise returns \term{false}.

\Defgeneric {find-child-output-record} {record use-old-elements record-type
                                        \rest initargs
                                        \key unique-id unique-id-test}

Finds a child of \arg{record} matching the \arg{record-type} and the supplied
initargs \arg{initargs}.  \arg{unique-id} and \arg{unique-id-test} are used to
match against the children as well.  \arg{use-old-elements} controls whether the
desired record is to be found in the previous (before redisplay) contents of the
record.

\Defgeneric {output-record-contents-ok} {record}

Returns \term{true} if the current state of \arg{record} are up to date,
otherwise returns \term{false}.

\Defgeneric {recompute-contents-ok} {record}

Compares the old (before redisplay) and new contents of \arg{record} to
determine whether or not this record changed in such a way so that the display
needs updating.

\Defgeneric {cache-output-record} {record child unique-id}

\arg{record} stores \arg{child} such that it can be located later using
\arg{unique-id}.

\Defgeneric {decache-child-output-record} {record child use-old-elements}

Invalidates the redisplay state of \arg{record}.

\Defgeneric {find-cached-output-record} {record use-old-elements record-type
                                         \rest initargs
                                         \key unique-id unique-id-test \allow}

Finds a previously cached child matching \arg{record-type}, \arg{initargs},
\arg{unique-id}, and \arg{unique-id-test}.  \arg{use-old-elements} controls
whether the desired record is to be found in the previous (before redisplay)
contents of the record.


\section {Incremental Redisplay Stream Protocol}

\Defgeneric {redisplayable-stream-p} {stream}

Returns \term{true} for any stream that maintains an output history and supports
the incremental redisplay protocol, otherwise returns \term{false}.

\Defgeneric {stream-redisplaying-p} {stream}

Returns \term{true} if the \arg{stream} is currently doing redisplay (that is,
is inside of a call to \cl{redisplay}), otherwise returns \term{false}.

\Defgeneric {incremental-redisplay} {stream position
                                     erases moves draws erase-overlapping move-overlapping} 

Performs the incremental update on \arg{stream} according to the difference set
comprised by \arg{erases}, \arg{moves}, \arg{draws}, \arg{erase-overlapping},
and \arg{move-overlapping}, which are values returned by
\cl{compute-difference-set}.  \arg{position} is a point object that represents
the start position of the topmost output record that will be redisplayed.

\cl{incremental-redisplay} can be called on any extended output stream.
