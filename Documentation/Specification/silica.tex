% -*- Mode: LaTeX; Package: CLIM-USER -*-

\chapter {Overview of Window Facilities}

\section {Introduction}

A central notion in organizing user interfaces is allocating screen regions to
particular tasks and recursively subdividing these regions into subregions.  The
windowing layer of CLIM defines an extensible framework for constructing, using,
and managing such \concept{hierarchies of interactive regions}.  This framework
allows uniform treatment of the following things:

\begin{itemize}
\item Window objects like those in X or NeWS.

\item Lightweight gadgets typical of toolkit layers, such as Motif or OpenLook.

\item Structured graphics like output records and an application's presentation
objects.

\item Objects that act as Lisp handles for windows or gadgets implemented in a
different language (such as OpenLook gadgets implemented in C).
\end{itemize}

From the perspective of most CLIM users, CLIM's windowing layer plays the role
of a window system.  However, CLIM will usually use the services of a window
system platform to provide efficient windowing, input, and output facilities.
In this specification, such window system platforms will be referred to as host
window systems or as display servers.

The fundamental window abstraction defined by CLIM is called a \concept{sheet}.
A sheet can participate in a relationship called a \concept{windowing
relationship}.  This relationship is one in which one sheet called the
\concept{parent} provides space to a number of other sheets called
\concept{children}.  Support for establishing and maintaining this kind of
relationship is the essence of what window systems provide.  At any point in
time, CLIM allows a sheet to be a child in one relationship called its
\concept{youth windowing relationship} and a parent in another relationship
called its \concept{adult windowing relationship}.

Programmers can manipulate unrooted hierarchies of sheets (those without a
connection to any particular display server).  However, a sheet hierarchy must
be attached to a display server to make it visible.  \concept{Ports} and
\concept{grafts} provide the functionality for managing this capability.  A
\term{port} is an abstract connection to a display service that is responsible
for managing host display server resources and for processing input events
received from the host display server.  A \term{graft} is a special kind of
sheet that represents a host window, typically a root window (that is, a
screen-level window).  A sheet is attached to a display by making it a child of
a graft, which represents an appropriate host window.  The sheet will then
appear to be a child of that host window.  So, a sheet is put onto a particular
screen by making it a child of an appropriate graft and enabling it.  Ports and
grafts are described in detail in Chapter~\ref{ports-and-grafts}.


\section {Properties of Sheets}

Sheets have the following properties:

\begin{description}
\item [A coordinate system] Provides the ability to refer to locations in a
sheet's abstract plane.

\item [A region] Defines an area within a sheet's coordinate system that
indicates the area of interest within the plane, that is, a clipping region for
output and input.  This typically corresponds to the visible region of the sheet
on the display.

\item [A parent] A sheet that is the parent in a windowing relationship in which
this sheet is a child.

\item [Children] An ordered set of sheets that are each a child in a windowing
relationship in which this sheet is a parent.  The ordering of the set
corresponds to the stacking order of the sheets.  Not all sheets have children.

\item [A transformation] Determines how points in this sheet's coordinate system
are mapped into points in its parents.

\item [An enabled flag] Indicates whether the sheet is currently actively
participating in the windowing relationship with its parent and siblings.

\item [An event handler] A procedure invoked when the display server wishes to
inform CLIM of external events.

\item [Output state] A set of values used when CLIM causes graphical or textual
output to appear on the display.  This state is often represented by a medium.
\end{description}


\section {Sheet Protocols}

A sheet is a participant in a number of protocols.  Every sheet must provide
methods for the generic functions that make up these protocols.  These protocols
are:

\begin{description}
\item [The windowing protocol] Describes the relationships between the sheet and
its parent and children (and, by extension, all of its ancestors and
descendants).

\item [The input protocol] Provides the event handler for a sheet.  Events may
be handled synchronously, asynchronously, or not at all.

\item [The output protocol] Provides graphical and textual output, and manages
descriptive output state such as color, transformation, and clipping.

\item [The repaint protocol] Invoked by the event handler and by user programs
to ensure that the output appearing on the display device appears as the program
expects it to appear.

\item [The notification protocol] Invoked by the event handler and user programs
to ensure that CLIM's representation of window system information is equivalent
to the display server's.
\end{description}

These protocols may be handled directly by a sheet, queued for later processing
by some other agent, or passed on to a delegate sheet for further processing.


\chapter {Properties of Sheets}
\label {sheet-properties}

\section {Basic Sheet Classes}

Note that there are no standard sheet classes in CLIM, and no pre-packaged way
to create sheets in general.  If a programmer needs to create an instance of
some class of sheet, \cl{make-instance} must be used.  For most purposes,
calling \cl{make-pane} is how application programmers will make sheets.

\Defprotoclass {sheet}

The protocol class that corresponds to a sheet.  This and the next chapter
describe all of the sheet protocols.
\IfYouWantClass {a} {sheet} {sheet}

All of the subclasses of \cl{sheet} are mutable.

\Defpredicate {sheetp} {object}

Returns \term{true} if \arg{object} is a \term{sheet}, otherwise returns
\term{false}.

\Defclass {basic-sheet}

The basic class on which all CLIM sheets are built, a subclass of \cl{sheet}.
\AbstractClass



\section {Relationships Between Sheets}

Sheets are arranged in a tree-structured, acyclic, top-down hierarchy.  Thus, in
general, a sheet has one parent (or no parent) and zero or more children.  A
sheet may have zero or more siblings (that is, other sheets that share the same
parent).  In order to describe the relationships between sheets, we need to
define some terms.

\begin{description}
\item [Adopted] A sheet is said to be \concept{adopted} if it has a parent.  A
sheet becomes the parent of another sheet by adopting that sheet.

\item [Disowned] A sheet is said to be \concept{disowned} if it does not have a
parent.  A sheet ceases to be a child of another sheet by being disowned.

\item [Grafted] A sheet is said to be \concept{grafted} when it is part of a
sheet hierarchy whose highest ancestor is a graft.  In this case, the sheet may
be visible on a particular window server.

\item [Degrafted] A sheet is said to be \concept{degrafted} when it is part of a
sheet hierarchy that cannot possibly be visible on a server, that is, the
highest ancestor is not a graft.

\item [Enabled] A sheet is said to be \concept{enabled} when it is actively
participating in the windowing relationship with its parent.  If a sheet is
enabled and grafted, and all its ancestors are enabled (they are grafted by
definition), then the sheet will be visible if it occupies a portion of the
graft region that isn't clipped by its ancestors or ancestors' siblings.

\item [Disabled] The opposite of enabled is \concept{disabled}.
\end{description}


\subsection {Sheet Relationship Functions}

The generic functions in this section comprise the sheet protocol.  All sheet
objects must implement or inherit methods for each of these generic functions.

\Defgeneric {sheet-parent} {sheet} 

Returns the parent of the \term{sheet} \arg{sheet}, or \cl{nil} if the sheet has
no parent.

\Defgeneric {sheet-children} {sheet}

Returns a list of sheets that are the children of the \term{sheet} \arg{sheet}.
Some sheet classes support only a single child; in this case, the result of
\cl{sheet-children} will be a list of one element.
\ReadOnly

\Defgeneric {sheet-adopt-child} {sheet child}

Adds the child sheet \arg{child} to the set of children of the \term{sheet}
\arg{sheet}, and makes the \arg{sheet} the child's parent.  If \arg{child}
already has a parent, the \cl{sheet-already-has-parent} error will be signalled.

Some sheet classes support only a single child.  For such sheets, attempting to
adopt more than a single child will cause the \cl{sheet-supports-only-one-child}
error to be signalled.

\Defgeneric {sheet-disown-child} {sheet child \key (errorp \cl{t})}

Removes the child sheet \arg{child} from the set of children of the \term{sheet}
\arg{sheet}, and makes the parent of the child be \cl{nil}.  If \arg{child} is
not actually a child of \arg{sheet} and \arg{errorp} is \term{true}, then the
\cl{sheet-is-not-child} error will be signalled.

\Defgeneric {sheet-siblings} {sheet}

Returns a list of all of the siblings of the \term{sheet} \arg{sheet}.  The
sibling are all the children of \arg{sheet}'s parent excluding \arg{sheet}
itself.
\FreshOutputs

\Defgeneric {sheet-enabled-children} {sheet}

Returns a list of those children of the \term{sheet} \arg{sheet} that are enabled.
\FreshOutputs

\Defgeneric {sheet-ancestor-p} {sheet putative-ancestor}

Returns \term{true} if the \term{sheet} \arg{putative-ancestor} is in fact
an ancestor of the \term{sheet} \arg{sheet}, otherwise returns \term{false}.

\defgeneric {raise-sheet} {sheet}
\Defgeneric {bury-sheet}  {sheet}

These functions reorder the children of a sheet by raising the \term{sheet}
\arg{sheet} to the top or burying it at the bottom.  Raising a sheet puts it at
the beginning of the ordering; burying it puts it at the end.  If sheets
overlap, the one that appears ``on top'' on the display device is earlier in the
ordering than the one underneath.

This may change which parts of which sheets are visible on the display device.

\Defgeneric {reorder-sheets} {sheet new-ordering}

Reorders the children of the \term{sheet} \arg{sheet} to have the new ordering
specified by \arg{new-ordering}.  \arg{new-ordering} is an ordered list of the
child sheets; elements at the front of \arg{new-ordering} are ``on top'' of
elements at the rear.

If \arg{new-ordering} does not contain all of the children of \arg{sheet}, the
\cl{sheet-ordering-underspecified} error will be signalled.  If
\arg{new-ordering} contains a sheet that is not a child of \arg{sheet}, the
\cl{sheet-is-not-child} error will be signalled.

\Defgeneric {sheet-enabled-p} {sheet}

Returns \term{true} if the \term{sheet} \arg{sheet} is enabled by its
parent, otherwise returns \term{false}.  Note that all of a sheet's ancestors
must be enabled before the sheet is viewable.

\Defgeneric {(setf sheet-enabled-p)} {enabled-p sheet}

When \arg{enabled-p} is \term{true}, this enables the \term{sheet}
\arg{sheet}.  When \arg{enabled-p} is \term{false}, this disables the sheet.

Note that a sheet is not visible unless it and all of its ancestors are enabled.


\Defgeneric {sheet-viewable-p} {sheet}

Returns \term{true} if the \term{sheet} \arg{sheet} and all its ancestors are
enabled, and if one of its ancestors is a graft.  See
Chapter~\ref{ports-and-grafts} for further information.

\Defgeneric {sheet-occluding-sheets} {sheet child}

Returns a list of the \term{sheet} \arg{child}'s siblings that occlude part or
all of the region of the \arg{child}.  In general, these are the siblings that
are enabled and appear earlier in the \term{sheet} \arg{sheet}'s children.  If
\arg{sheet} does not permit overlapping among its children,
\cl{sheet-occluding-sheets} will return \cl{nil}.

\FreshOutputs

\Defgeneric {map-over-sheets} {function sheet}

Applies the function \arg{function} to the sheet \arg{sheet}, and then applies
\arg{function} to all of the descendents (the children, the children's children,
and so forth) of \arg{sheet}.

Function is a function of one argument, the sheet; it has dynamic extent.


\subsection {Sheet Genealogy Classes}

Different ``mixin'' classes are provided that implement the relationship
protocol.  None of the four following classes are instantiable.

\Defclass {sheet-parent-mixin}

This class is mixed into sheet classes that have a parent.

\Defclass {sheet-leaf-mixin}

This class is mixed into sheet classes that will never have children.

\Defclass {sheet-single-child-mixin}

This class is mixed into sheet classes that have at most a single child.

\Defclass {sheet-multiple-child-mixin}

This class is mixed into sheet classes that may have zero or more children.


\section {Sheet Geometry}

Every sheet has a region and a coordinate system.  A sheet's region refers to
its position and extent on the display device, and is represented by some sort
of a region object, frequently a rectangle.  A sheet's coordinate system is
represented by a coordinate transformation that converts coordinates in its
coordinate system to coordinates in its parent's coordinate system.

\subsection {Sheet Geometry Functions}

\defgeneric {sheet-transformation} {sheet}
\Defgeneric {(setf sheet-transformation)} {transformation sheet}

Returns a transformation that converts coordinates in the \term{sheet}
\arg{sheet}'s coordinate system into coordinates in its parent's coordinate
system.  Using \cl{setf} on this accessor will modify the sheet's coordinate
system, including moving its region in its parent's coordinate system.

When the sheet's transformation is changed,
\cl{note-sheet-transformation-changed} is called on the \arg{sheet} to notify the sheet of
the change.

\defgeneric {sheet-region} {sheet}
\Defgeneric {(setf sheet-region)} {region sheet}

Returns a region object that represents the set of points to which the
\term{sheet} \arg{sheet} refers.  The region is in the sheet's coordinate
system.  Using \cl{setf} on this accessor modifies the sheet's region.

When the sheet's region is changed, \cl{note-sheet-region-changed} is called on
\arg{sheet} to notify the sheet of the change.

\issue {RSL} {To reshape and move a region, you generally have to manipulate
both of the above.  Maybe there should be a single function that takes either
or both of a new transformation or region?  Maybe region coordinates should be
expressed in parents' coordinates, since that's easier to set only one?}

\issue {RSL} {I'm not convinced I like this business of requesting a change by
modifying an accessor.  It might be better to have a \cl{request-} function, so
it would be clear that there might be some delay before the region or
transformation was modified.  Currently, using \cl{setf} on mirrored sheets
requests that the server move or resize the sheet; the accessor will continue to
return the old value until the notification comes in from the display server
that says that the mirror has been moved.}


\Defgeneric {move-sheet} {sheet x y}

Moves the \term{sheet} \arg{sheet} to the new position $(x,y)$.  \arg{x} and
\arg{y} are expressed in the coordinate system of \arg{sheet}'s parent.

\cl{move-sheet} simply modifies \arg{sheet}'s transformation, and could be
implemented as follows:

\begin{verbatim}
(defmethod move-sheet ((sheet basic-sheet) x y)
  (let ((transform (sheet-transformation sheet)))
    (multiple-value-bind (old-x old-y)
        (transform-position transform 0 0)
      (setf (sheet-transformation sheet)
            (compose-translation-with-transformation
              transform (- x old-x) (- y old-y))))))
\end{verbatim}

\Defgeneric {resize-sheet} {sheet width height}

Resizes the \term{sheet} \arg{sheet} to have a new width \arg{width} and a new
height \arg{height}.  \arg{width} and \arg{height} are real numbers.

\cl{resize-sheet} simply modifies \arg{sheet}'s region, and could be implemented
as follows:

\begin{verbatim}
(defmethod resize-sheet ((sheet basic-sheet) width height)
  (setf (sheet-region sheet)
        (make-bounding-rectangle 0 0 width height)))
\end{verbatim}


\Defgeneric {move-and-resize-sheet} {sheet x y width height}

Moves the \term{sheet} \arg{sheet} to the new position $(x,y)$, and changes its
size to the new width \arg{width} and the new height \arg{height}.  \arg{x} and
\arg{y} are expressed in the coordinate system of \arg{sheet}'s parent.
\arg{width} and \arg{height} are real numbers.

\cl{move-and-resize-sheet} could be implemented as follows:

\begin{verbatim}
(defmethod move-and-resize-sheet ((sheet basic-sheet) x y width height)
  (move-sheet sheet x y)
  (resize-sheet sheet width height))
\end{verbatim}


\Defgeneric {map-sheet-position-to-parent} {sheet x y}

Applies the \term{sheet} \arg{sheet}'s transformation to the point $(x,y)$,
returning the coordinates of that point in \arg{sheet}'s parent's coordinate
system.

\Defgeneric {map-sheet-position-to-child} {sheet x y}

Applies the inverse of the \term{sheet} \arg{sheet}'s transformation to the
point $(x,y)$ (represented in \arg{sheet}'s parent's coordinate system),
returning the coordinates of that same point in \arg{sheet} coordinate system.

\Defgeneric {map-sheet-rectangle*-to-parent} {sheet x1 y1 x2 y2} 

Applies the \term{sheet} \arg{sheet}'s transformation to the bounding rectangle
specified by the corner points $(x1,y1)$ and $(x2,y2)$, returning the bounding
rectangle of the transformed region as four values, \arg{min-x}, \arg{min-y},
\arg{max-x}, and \arg{max-y}.  The arguments \arg{x1}, \arg{y1}, \arg{x2}, and
\arg{y1} are canonicalized in the same way as for \cl{make-bounding-rectangle}.

\Defgeneric {map-sheet-rectangle*-to-child} {sheet x1 y1 x2 y2}

Applies the inverse of the \term{sheet} \arg{sheet}'s transformation to the
bounding rectangle delimited by the corner points $(x1,y1)$ and $(x2,y2)$
(represented in \arg{sheet}'s parent's coordinate system), returning the
bounding rectangle of the transformed region as four values, \arg{min-x},
\arg{min-y}, \arg{max-x}, and \arg{max-y}.  The arguments \arg{x1}, \arg{y1},
\arg{x2}, and \arg{y2} are canonicalized in the same way as for
\cl{make-bounding-rectangle}.

\issue {SWM} {I now think that \cl{map-} in these names is misleading; maybe
\cl{convert-} is better?}

\Defgeneric {map-over-sheets-containing-position} {function sheet x y}

Applies the function \arg{function} to all of the children of the sheet
\arg{sheet} that contain the position $(x,y)$.  \arg{x} and \arg{y} are
expressed in \arg{sheet}'s coordinate system.

Function is a function of one argument, the sheet; it has dynamic extent.

\Defgeneric {map-over-sheets-overlapping-region} {function sheet region}

Applies the function \arg{function} to all of the children of the sheet
\arg{sheet} that overlap the region \arg{region}.  \arg{region} is expressed in
\arg{sheet}'s coordinate system.

Function is a function of one argument, the sheet; it has dynamic extent.


\Defgeneric {child-containing-position} {sheet x y}

Returns the topmost enabled direct child of the \term{sheet} \arg{sheet} whose
region contains the position $(x,y)$.  The position is expressed in
\arg{sheet}'s coordinate system.

\defgeneric {children-overlapping-region}     {sheet region}
\Defgeneric {children-overlapping-rectangle*} {sheet x1 y1 x2 y2}

Returns the list of enabled direct children of the \term{sheet} \arg{sheet}
whose region overlaps the \term{region} \arg{region}.
\cl{children-overlapping-rectangle*} is a special case of
\cl{children-overlapping-region} in which the region is a bounding rectangle
whose corner points are $(x1,y1)$ and $(x2,y2)$.  The region is expressed in
\arg{sheet}'s coordinate system.
\FreshOutputs


\Defgeneric {sheet-delta-transformation} {sheet ancestor}

Returns a transformation that is the composition of all of the sheet
transformations between the \term{sheets} \arg{sheet} and \arg{ancestor}.  If
\arg{ancestor} is \cl{nil}, \cl{sheet-delta-transformation} will return the
transformation to the root of the sheet hierarchy.  If \arg{ancestor} is not an
ancestor of sheet, the \cl{sheet-is-not-ancestor} error will be signalled.

The computation of the delta transformation is likely to be cached.

\Defgeneric {sheet-allocated-region} {sheet child}

Returns the visible region of the \term{sheet} \arg{child} in the \term{sheet}
\arg{sheet}'s coordinate system.  If \arg{child} is occluded by any of its
siblings, those siblings' regions are subtracted (using \cl{region-difference})
from \arg{child}'s actual region.


\subsection {Sheet Geometry Classes}

Each of the following implements the sheet geometry protocol in a different
manner, according to the sheet's requirements.  None of the four following
classes are instantiable.

\Defclass {sheet-identity-transformation-mixin}

This class is mixed into sheet classes whose coordinate system is identical to
that of its parent.

\Defclass {sheet-translation-mixin}

This class is mixed into sheet classes whose coordinate system is related to
that of its parent by a simple translation.

\Defclass {sheet-y-inverting-transformation-mixin}

This class is mixed into sheet classes whose coordinate system is related to
that of its parent by inverting the $y$ coordinate system, and optionally
translating by some amount in $x$ and $y$.

\Defclass {sheet-transformation-mixin}

This class is mixed into sheet classes whose coordinate system is related to
that of its parent by an arbitrary affine transformation.  CLIM implementations
are allowed to restrict these transformations to just rectilinear ones.


\chapter {Sheet Protocols}
\label {sheet-protocols}

\section {Input Protocol}

CLIM's windowing substrate provides an input architecture and standard
functionality for notifying clients of input that is distributed to their
sheets.  Input includes such events as the pointer entering and exiting sheets,
pointer motion (whose granularity is defined by performance limitations), and
pointer button and keyboard events.  At this level, input is represented as
\term{event} objects.

Sheets either participate fully in the input protocol or are mute for input.  If
any functions in the input protocol are called on a sheet that is mute for
input, the \cl{sheet-is-mute-for-input} error will be signalled.

In addition to handling input events, a sheet is also responsible for providing
other input services, such as controlling the pointer's appearance, and polling
for current pointer and keyboard state.

Input is processed on a per-port basis by the function \cl{process-next-event}.
In multiprocessing environments, a process that calls \cl{process-next-event} in a
loop is created for each port.  In single-process Lisps, \cl{process-next-event}
is called whenever the user would go blocked for input.

\cl{process-next-event} has three main tasks when it receives an event.  First,
it must determine to which \concept{client} the event is addressed; this process
is called \concept{distributing}.  Typically, the client is a sheet, but there
are other special-purpose clients to which events can also be dispatched.  Next,
\cl{process-next-event} formats the event into a standard format, and finally it
\concept{dispatches} the event to the client.  A client may then either handle
the event synchronously, or it may queue it for later handling by another
process.

Input events can be broadly categorized into \concept{pointer events} and
\concept{keyboard events}.  By default, pointer events are dispatched to the
lowest sheet in the hierarchy whose region contains the location of the pointer.
Keyboard events are dispatched to the port's keyboard input focus; the accessor
\cl{port-keyboard-input-focus} contains the event client that receives the
port's keyboard events.


\subsection {Input Protocol Functions}

In the functions below, the \arg{client} argument is typically a sheet, but it
may be another object that supports event distribution, dispatching, and
handling.

\Defgeneric {sheet-event-queue} {sheet}

Any sheet that can process events will have an event queue from which the events
are gotten.  \cl{sheet-event-queue} returns the object that acts as the event
queue.  The exact representation of an event queue is explicitly unspecified.

\Defgeneric {process-next-event} {port \key wait-function timeout}

This function provides a standard interface for one pass through a
port's event processing loop. \arg{wait-function} is either \cl{nil}
or a function of no arguments that acts as a predicate; it has dynamic
extent. \cl{process-next-event} should wait until one of three
conditions occurs:

\begin{itemize}
\item If an event was received and processed, the predicate should return \term{true}.
\item If a timeout occurs, the function should return the two values \term{false} and \cl{:timeout}.
\item If the wait function returns \term{true}, the function should return the two values \term{false} and \cl{:wait-function}.
\end{itemize}

A port implementation must provide a method for this function that
reads the next window server-specific device event, blocking if
necessary, and then invokes the event distributor.

\issue {DK} {When \arg{wait-function} should be called? Either we busy
  wait for the predicate returning true (and compute timeout decay on
  our own) or call wait-function after some input for port has been
  processed (then wait-function won't be able to notice asynchronous
  changes to the state it tries to monitor).}

\defgeneric {port-keyboard-input-focus} {port}
\Defgeneric {(setf port-keyboard-input-focus)} {focus port}

Returns the client to which keyboard events are to be dispatched.

\issue {RSL} {Should this accessor be called \cl{keyboard-input-focus}?  It may
be that we want to be able to call it on a frame in order to implement some sort
of per-frame input focus.}

\Defgeneric {distribute-event} {port event}

The \arg{event} is distributed to the \arg{port}'s proper client.  In general,
this will be the keyboard input focus for keyboard events, and the lowest sheet
under the pointer for pointer events.

\issue {RSL} {Do we just want to call this function \cl{dispatch-event} and have
it called on the port first?}

\Defgeneric {dispatch-event} {client event}

This function is called by \cl{process-next-event} to inform a client about an
event of interest.  It is invoked synchronously by whatever process called
\cl{process-next-event}, so many methods for this function will simply queue the
event for later handling.  Certain classes of clients and events may cause this
function immediately to call either \cl{queue-event} or \cl{handle-event}, or to
ignore the event entirely.

\Defgeneric {queue-event} {client event}

Places the event \arg{event} into the queue of events for the client \arg{client}.

\Defgeneric {handle-event} {client event}

Implements the client's policy with respect to the event.  For example, if the
programmer wishes to highlight a sheet in response to an event that informs it
that the pointer has entered its territory, there would be a method to carry out
the policy that specializes the appropriate sheet and event classes.

In addition to \cl{queue-event}, the queued input protocol handles the following
generic functions.  The \arg{client} argument to these functions is typically a
sheet.

\Defgeneric {event-read} {client}

Takes the next event out of the queue of events for this client.

\Defgeneric {event-read-no-hang} {client}

Takes the next event out of the queue of events for this client.  It returns
\cl{nil} if there are no events in the queue.

\Defgeneric {event-peek} {client \optional event-type}

Returns the next event in the queue without removing it from the queue.  If
\arg{event-type} is supplied, events that are not of that type are first
removed and discarded.

\Defgeneric {event-unread} {client event}

Places the \arg{event} at the head of the \arg{client}'s event queue, so as to
be the next event read.

\Defgeneric {event-listen} {client}

Returns \term{true} if there are any events queued for \arg{client}, otherwise
returns \term{false}.


\subsection {Input Protocol Classes}

Most classes of sheets will have one of the following input protocol classes
mixed in.  Of course, a sheet can always have a specialized method for a
specific class of event that will override the default.  For example, a sheet
may need to have only pointer click events dispatched to itself, and may
delegate all other events to some other input client.  Such a sheet should have
\cl{delegate-sheet-input-mixin} as a superclass, and have a more specific method
for \cl{dispatch-event} on its class and \cl{pointer-button-click-event}.

None of the five following classes are instantiable.

\Defclass {standard-sheet-input-mixin}

This class of sheet provides a method for \cl{dispatch-event} that calls
\cl{queue-event} on each device event.  Note that configuration events invoke
\cl{handle-event} immediately.

The \cl{standard-sheet-input-mixin} class will also provide a
\cl{sheet-event-queue} method, described above.

\Defclass {immediate-sheet-input-mixin}

This class of sheet provides a method for \cl{dispatch-event} that calls
\cl{handle-event} immediately for all events.

The \cl{immediate-sheet-input-mixin} class will also provide a
\cl{sheet-event-queue} method, described above.

\Defclass {sheet-mute-input-mixin}

This is mixed in to any sheet class that does not handle any input events.

\Defclass {delegate-sheet-input-mixin}

This class of sheet provides a method for \cl{dispatch-event} that calls
\cl{dispatch-event} on a designated substitute recipient and the event.  The
initarg \cl{:delegate} or the accessor {delegate-sheet-delegate} may be used to
set the recipient of dispatched events.

\defgeneric {delegate-sheet-delegate} {sheet}
\Defgeneric {(setf delegate-sheet-delegate)} {delegate sheet}

This may be set to another recipient of events dispatched to a sheet of class
\cl{delegate-sheet-input-mixin}.  \arg{delegate} is the object to which events
will be dispatched, and is usually another sheet.  If the delegate is \cl{nil},
events are discarded.




\section {Standard Device Events}

An \concept{event} is a CLIM object that represents some sort of user gesture
(such as moving the pointer or pressing a key on the keyboard) or that
corresponds to some sort of notification from the display server.  Event objects
store such things as the sheet associated with the event, the $x$ and $y$
position of the pointer within that sheet, the key name or character
corresponding to a key on the keyboard, and so forth.

Figure~\ref{event-hier} shows all the event classes.

\begin{figure}
\hrule
\begin{tabbing}
XX\=XX\=XX\=XX\=XX\=XX\=  \kill
\cl{event} \\
\>\cl{device-event}          \\
\>\>\cl{keyboard-event}      \\
\>\>\>\cl{key-press-event}   \\
\>\>\>\cl{key-release-event} \\
\>\>\cl{pointer-event}       \\
\>\>\>\cl{pointer-button-event} \\
\>\>\>\>\cl{pointer-button-press-event}   \\
\>\>\>\>\cl{pointer-button-release-event} \\
\>\>\>\>\cl{pointer-button-hold-event}    \\
\>\>\>\cl{pointer-motion-event}  \\
\>\>\>\>\cl{pointer-boundary-event} \\
\>\>\>\>\>\cl{pointer-enter-event} \\
\>\>\>\>\>\cl{pointer-exit-event}  \\
\>\cl{window-event} \\
\>\>\cl{window-configuration-event} \\
\>\>\cl{window-repaint-event}       \\
\>\cl{window-manager-event}           \\
\>\>\cl{window-manager-delete-event} \\
\>\cl{timer-event} \\
\end{tabbing}
\caption{\label{event-hier} CLIM event classes.  All classes that appear
at a given indentation are subclasses of the class that appears above and at a
lesser indentation.} 
\vspace{2pc}
\hrule
\end{figure}

\Defprotoclass {event}

The protocol class that corresponds to any sort of ``event''.
\IfYouWantClass {an} {event} {event}

All of the event classes are immutable.  CLIM implementations may choose to keep
a resource of the device event classes, but this must be invisible at the API
level.  That is, any event visible at the level of the API must act as though it
is immutable.

\Defpredicate {eventp} {object}

Returns \term{true} if \arg{object} is an \term{event}, otherwise returns
\term{false}.

\Definitarg {:timestamp}

All subclasses of \cl{event} must take a \cl{:timestamp} initarg, which is used
to specify the timestamp for the event.

\Defgeneric {event-timestamp} {event}

Returns an integer that is a monotonically increasing timestamp for the
\term{event} \arg{event}.  The timestamp must have at least as many bits of
precision as a fixnum.

\Defgeneric {event-type} {event}

For the \term{event} \arg{event}, returns a keyword with the same name as the
class name, except stripped of the ``-event'' ending.  For example, the keyword
\cl{:key-press} is returned by \cl{event-type} for an event whose class is
\cl{key-press-event}.

All event classes must implement methods for \cl{event-type} and
\cl{event-timestamp}.


\defclass {device-event}
\definitarg {:sheet}
\Definitarg {:modifier-state}

The instantiable class that corresponds to any sort of device event.  This is a
subclass of \cl{event}.

All subclasses of \cl{device-event} must take the \cl{:sheet} and
\cl{:modifier-state} initargs, which are used to specify the sheet and modifier
state components for the event.

\Defgeneric {event-sheet} {device-event}

Returns the sheet associated with the event \arg{device-event}.

\Defgeneric {event-modifier-state} {device-event}

Returns an integer value that encodes the state of all the modifier keys on the
keyboard.  This will be a mask consisting of the \cl{logior} of
\cl{+shift-key+}, \cl{+control-key+}, \cl{+meta-key+}, \cl{+super-key+}, and
\cl{+hyper-key+}.

All device event classes must implement methods for \cl{event-sheet} and
\cl{event-modifier-state}.


\defclass {keyboard-event}
\Definitarg {:key-name}

The instantiable class that corresponds to any sort of keyboard event.  This is
a subclass of \cl{device-event}.

All subclasses of \cl{keyboard-event} must take the \cl{:key-name} initarg,
which is used to specify the key name component for the event.

\Defgeneric {keyboard-event-key-name} {keyboard-event}

Returns the name of the key that was pressed or released in a keyboard event.
This will be a symbol whose value is port-specific.  Key names corresponding to
the set of ``standard'' characters (such as the alphanumerics) will be a symbol
in the keyword package.

\Defgeneric {keyboard-event-character} {keyboard-event}

Returns the character associated with the event \arg{keyboard-event}, if there
is any.

All keyboard event classes must implement methods for
\cl{keyboard-event-key-name} and \cl{keyboard-event-character}.

\defclass {key-press-event}
\Defclass {key-release-event}
 
The instantiable classes that correspond to a key press or release event.  This
is a subclass of \cl{keyboard-event}.


\defclass {pointer-event}
\definitarg {:pointer}
\definitarg {:button}
\definitarg {:x}
\Definitarg {:y}
 
The instantiable class that corresponds to any sort of pointer event.  This is a
subclass of \cl{device-event}.

All subclasses of \cl{pointer-event} must take the \cl{:pointer}, \cl{:button},
\cl{:x}, and \cl{:y} initargs, which are used to specify the pointer object,
pointer button, and native $x$ and $y$ position of the pointer at the time of
the event.  The sheet's $x$ and $y$ positions are derived from the supplied
native $x$ and $y$ positions and the sheet itself.

\defgeneric {pointer-event-x} {pointer-event}
\Defgeneric {pointer-event-y} {pointer-event}

Returns the $x$ and $y$ position of the pointer at the time the event occurred,
in the coordinate system of the sheet that received the event.  All pointer
events must implement a method for these generic functions.

\defgeneric {pointer-event-native-x} {pointer-event}
\Defgeneric {pointer-event-native-y} {pointer-event}

Returns the $x$ and $y$ position of the pointer at the time the event occurred,
in the pointer's native coordinate system.  All pointer events must implement a
method for these generic functions.

\Defgeneric {pointer-event-pointer} {pointer-event}

Returns the pointer object to which this event refers.

All pointer event classes must implement methods for \cl{pointer-event-x},
\cl{pointer-event-y}, \cl{pointer-event-native-x}, \cl{pointer-event-native-y},
and \cl{pointer-event-pointer}.

\Defclass {pointer-button-event}
 
The instantiable class that corresponds to any sort of pointer button event.
This is a subclass of \cl{pointer-event}.

\Defgeneric {pointer-event-button} {pointer-button-event}

Returns the an integer corresponding to the pointer button that was pressed or
released, which will be one of \cl{+pointer-left-button+},
\cl{+pointer-middle-button+}, or \cl{+pointer-right-button+}.

\defclass {pointer-button-press-event}
\defclass {pointer-button-release-event}
\Defclass {pointer-button-hold-event}
 
The instantiable classes that correspond to a pointer button press, button
release, and click-and-hold events.  These are subclasses of
\cl{pointer-button-event}.

\defclass {pointer-click-event}
\defclass {pointer-double-click-event}
\Defclass {pointer-click-and-hold-event}
 
The instantiable classes that correspond to a pointer button press, followed
immediately by (respectively) a button release, another button press, or pointer
motion.  These are subclasses of \cl{pointer-button-event}.  Ports are not
required to generate these events.

\Defclass {pointer-motion-event}
 
The instantiable class that corresponds to any sort of pointer motion event.
This is a subclass of \cl{pointer-event}.

\Defclass {pointer-boundary-event}

The instantiable class that corresponds to a pointer motion event that crosses
some sort of sheet boundary.  This is a subclass of \cl{pointer-motion-event}.

\Defgeneric {pointer-boundary-event-kind} {pointer-boundary-event}

Returns the ``kind'' of boundary event, which will be one of \cl{:ancestor},
\cl{:virtual}, \cl{:inferior}, \cl{:nonlinear}, \cl{:nonlinear-virtual}, or
\cl{nil}.  These event kinds correspond to the detail members for X11 enter and
exit events.

\defclass {pointer-enter-event}
\Defclass {pointer-exit-event}

The instantiable classes that correspond to a pointer enter or exit event.
These are subclasses of \cl{pointer-boundary-event}.


\defclass {window-event}
\Definitarg {:region}

The instantiable class that corresponds to any sort of windowing event.  This is
a subclass of \cl{event}.

All subclasses of \cl{window-event} must take a \cl{:region} initarg, which is
used to specify the damage region associated with the event.

\Defgeneric {window-event-region} {window-event}

Returns the region of the sheet that is affected by a window event.

\Defgeneric {window-event-native-region} {window-event}

Returns the region of the sheet in native coordinates.

\Defgeneric {window-event-mirrored-sheet} {window-event}

Returns the mirrored sheet that is attached to the mirror on which the event
occurred.

All window event classes must implement methods for \cl{window-event-region},
\cl{window-event-native-region}, and \cl{window-event-mirrored-sheet}.

\Defclass {window-configuration-event}

The instantiable class that corresponds to a window changing its size or
position.  This is a subclass of \cl{window-event}.

\Defclass {window-repaint-event}

The instantiable class that corresponds to a request to repaint the window.
This is a subclass of \cl{window-event}.


\defclass {window-manager-event}
\Definitarg {:sheet}

The instantiable class that corresponds to any sort of window manager event.
This is a subclass of \cl{event}.

All subclasses of \cl{window-manager-event} must take a \cl{:sheet} initarg,
which is used to specify the sheet on which the window manager acted.

\Defclass {window-manager-delete-event}

The instantiable class that corresponds to window manager event that causes a
host window to be deleted.  This is a subclass of \cl{window-manager-event}.


\Defclass {timer-event}

The instantiable class that corresponds to a timer event.  This is a subclass
of \cl{event}.


\defconst {+pointer-left-button+}
\defconst {+pointer-middle-button+}
\Defconst {+pointer-right-button+}

Constants that correspond to the left, middle, and right button on a pointing
device.  \cl{pointer-event-button} will returns one of these three values.

These constants are powers of 2 so that they can be combined with \cl{logior}
and tested with \cl{logtest}.

\defconst {+shift-key+}
\defconst {+control-key+}
\defconst {+meta-key+}
\defconst {+super-key+}
\Defconst {+hyper-key+}

Constants that correspond to the shift, control, meta, super, and hyper modifier
keys being held down on the keyboard.  These constants are powers of 2 that are
disjoint from the pointer button constants, so that they can be combined with
\cl{logior} and tested with \cl{logtest}.

\cl{event-modifier-state} will return some combination of these values.

Implementations must support at least shift, control, and meta modifiers.
Control and meta might correspond to the control and option or command shift
keys on a Macintosh keyboard, for example.


\section {Output Protocol}

The output protocol is concerned with the appearance of displayed output on the
window associated with a sheet.  The sheet output protocol is responsible for
providing a means of doing output to a sheet, and for delivering repaint
requests to the sheet's client.

Sheets either participate fully in the output protocol or are mute for output.
If any functions in the output protocol are called on a sheet that is mute for
output, the \cl{sheet-is-mute-for-output} error will be signalled.


\subsection {Output Properties}

Each sheet retains some output state that logically describes how output is to
be rendered on its window.  Such information as the foreground and background
ink, line thickness, and transformation to be used during drawing are provided
by this state.  This state may be stored in a \concept{medium} associated with
the sheet itself, be derived from a parent, or may have some global default,
depending on the sheet itself.

If a sheet is mute for output, it is an error to set any of these values.

\Defprotoclass {medium}

The protocol class that corresponds to the output state for some kind of sheet.
There is no single advertised standard medium class.
\IfYouWantClass {a} {medium} {medium}

\Defpredicate {mediump} {object}

Returns \term{true} if \arg{object} is a \term{medium}, otherwise returns
\term{false}.

\Defclass {basic-medium}

The basic class on which all CLIM mediums are built, a subclass of \cl{medium}.
\AbstractClass


The following generic functions comprise the basic medium protocol.  All mediums
must implement methods for these generic functions.  Often, a sheet class that
supports the output protocol will implement a ``trampoline'' method that passes
the operation on to \cl{sheet-medium} of the sheet.

\defgeneric {medium-foreground} {medium}
\Defgeneric {(setf medium-foreground)} {design medium}

Returns (and, with \cl{setf}, sets) the current foreground ink for the
\term{medium} \arg{medium}. This is described in detail in
Chapter~\ref{drawing-options}.

\defgeneric {medium-background} {medium}
\Defgeneric {(setf medium-background)} {design medium}

Returns (and, with \cl{setf}, sets) the current background ink for the
\term{medium} \arg{medium}.  This is described in detail in
Chapter~\ref{drawing-options}.


\defgeneric {medium-ink} {medium}
\Defgeneric {(setf medium-ink)} {design medium}

Returns (and, with \cl{setf}, sets) the current drawing ink for the
\term{medium} \arg{medium}.  This is described in detail in
Chapter~\ref{drawing-options}.


\defgeneric {medium-transformation} {medium}
\Defgeneric {(setf medium-transformation)} {transformation medium}

Returns (and, with \cl{setf}, sets) the ``user'' transformation that converts
the coordinates presented to the drawing functions by the programmer to the
\term{medium} \arg{medium}'s coordinate system. By default, it is the identity
transformation.  This is described in detail in Chapter~\ref{drawing-options}.


\defgeneric {medium-clipping-region} {medium}
\Defgeneric {(setf medium-clipping-region)} {region medium}

Returns (and, with \cl{setf}, sets) the clipping region that encloses all output
performed on the \term{medium} \arg{medium}.  It is returned and set in user
coordinates.  That is, to convert the user clipping region to medium
coordinates, it must be transformed by the value of \cl{medium-transformation}.
For example, the values returned by

\begin{verbatim}
(let (cr1 cr2)
  ;; Ensure that the sheet's clipping region and transformation will be reset:
  (with-drawing-options (sheet :transformation +identity-transformation+
                               :clipping-region +everywhere+)
    (setf (medium-clipping-region sheet) (make-rectangle* 0 0 10 10))
    (setf (medium-transformation sheet) (clim:make-scaling-transformation 2 2))
    (setf cr1 (medium-clipping-region sheet))
    (setf (medium-clipping-region sheet) (make-rectangle* 0 0 10 10))
    (setf (medium-transformation sheet) +identity-transformation+)
    (setf cr2 (medium-clipping-region sheet))
    (values cr1 cr2)))
\end{verbatim}

are two rectangles.  The first one has edges of (0,0,5,5), while the second one
has edges of (0,0,20,20).

By default, the user clipping region is the value of \cl{+everywhere+}.

\Issue {SWM} {What exactly are ``user coordinates''?  We need to define all of
the coordinate systems in one place: device, window, stream, etc.}


\defgeneric {medium-line-style} {medium}
\Defgeneric {(setf medium-line-style)} {line-style medium}

Returns (and, with \cl{setf}, sets) the current line style for the \term{medium}
\arg{medium}.  This is described in detail in Chapter~\ref{drawing-options} and
Section~\ref{line-styles}.


\defgeneric {medium-text-style} {medium}
\Defgeneric {(setf medium-text-style)} {text-style medium}

Returns (and, with \cl{setf}, sets) the current text style for the \term{medium}
\arg{medium} of any textual output that may be displayed on the window.  This is
described in detail in Chapter~\ref{drawing-options}.


\defgeneric {medium-default-text-style} {medium}
\Defgeneric {(setf medium-default-text-style)} {text-style medium}

Returns (and, with \cl{setf}, sets) the default text style for output on the
\term{medium} \arg{medium}.  This is described in detail in
Chapter~\ref{drawing-options}.


\Defgeneric {medium-merged-text-style} {medium}

Returns the actual text style used in rendering text on the \term{medium}
\arg{medium}.  It returns the result of

\begin{verbatim} 
(merge-text-styles (medium-text-style medium)
                   (medium-default-text-style medium))
\end{verbatim}

Thus, those components of the current text style that are not \cl{nil} will
replace the defaults from medium's default text style.  Unlike the preceding
text style function, \cl{medium-merged-text-style} is read-only.


\subsection {Output Protocol Functions} 

The output protocol functions on mediums (and sheets that support the
standard output protocol) include those functions described in
Section~\ref{graphics-protocols}.

\issue {SWM} {We need to do a little better than this.}


\subsection {Output Protocol Classes}

The following classes implement the standard output protocols.  None of the five
following classes are instantiable.

\Defclass {standard-sheet-output-mixin}

This class is mixed in to any sheet that provides the standard output protocol,
such as repainting and graphics.

\Defclass {sheet-mute-output-mixin}

This class is mixed in to any sheet that provides none of the output protocol.

\Defclass {sheet-with-medium-mixin}

This class is used for any sheet that has either a permanent or a temporary
medium.

\Defclass {permanent-medium-sheet-output-mixin}

This class is mixed in to any sheet that always has a medium associated with it.
It is a subclass of \cl{sheet-with-medium-mixin}.

\Defclass {temporary-medium-sheet-output-mixin}

This class is mixed in to any sheet that may have a medium associated with it,
but does not necessarily have a medium at any given instant.  It is a subclass
of \cl{sheet-with-medium-mixin}.


\subsection {Associating a Medium with a Sheet}

Before a sheet may be used for output, it must be associated with a medium.
Some sheets are permanently associated with media for output efficiency; for
example, CLIM window stream sheets have a medium that is permanently allocated
to the window.

However, many kinds of sheets only perform output infrequently, and therefore do
not need to be associated with a medium except when output is actually required.
Sheets without a permanently associated medium can be much more lightweight than
they otherwise would be.  For example, in a program that creates a sheet for the
purpose of displaying a border for another sheet, the border sheet receives
output only when the window's shape is changed.

To associate a sheet with a medium, the macro \cl{with-sheet-medium} is used.
Only sheets that are subclasses of \cl{sheet-with-medium-mixin} may have a
medium associated with them.


\Defmacro {with-sheet-medium} {(medium sheet) \body body}

Within the body, the variable \arg{medium} is bound to the sheet's medium.  If
the sheet does not have a medium permanently allocated, one will be allocated
and associated with the sheet for the duration of the body (by calling
\cl{engraft-medium}), and then degrafted from the sheet and deallocated when the
body has been exited.  The values of the last form of the body are returned as
the values of \cl{with-sheet-medium}.

This macro will signal a runtime error if sheet is not a subclass of
\cl{sheet-with-medium-mixin}.

The \arg{medium} argument is not evaluated, and must be a symbol that will be bound
to a medium.  \arg{body} may have zero or more declarations as its first forms.

\Defmacro {with-sheet-medium-bound} {(sheet medium) \body body}

\cl{with-sheet-medium-bound} is used to associate the specific medium
\arg{medium} with the sheet \arg{sheet} for the duration of the body \arg{body}.
Typically, a single medium will be allocated and passed to several different
sheets that can use the same medium.

If the sheet already has a medium allocated to it, the new medium will not be
grafted to the sheet, and \cl{with-sheet-medium-bound} will simple evaluate the
body.  If the value of \arg{medium} is \cl{nil}, \cl{with-sheet-medium-bound} is
exactly equivalent to \cl{with-sheet-medium}.  The values of the last form of
the body are returned as the values of \cl{with-sheet-medium-bound}.

This macro will signal a runtime error if sheet is not a subclass of
\cl{sheet-with-medium-mixin}.

\arg{body} may have zero or more declarations as its first forms.

\Defgeneric {sheet-medium} {sheet}

Returns the medium associated with the \term{sheet} \arg{sheet}.  If \arg{sheet}
does not have a medium allocated to it, \cl{sheet-medium} returns \cl{nil}.

This function will signal an error if sheet is not a subclass of
\cl{sheet-with-medium-mixin}.

\Defgeneric {medium-sheet} {medium}

Returns the sheet associated with the \term{medium} \arg{medium}.  If
\arg{medium} is not grafted to a sheet, \cl{medium-sheet} returns \cl{nil}.

\Defgeneric {medium-drawable} {medium}

Returns an implementation-dependent object that corresponds to the actual host
window that will be drawn on when the \term{medium} \arg{medium} is drawn on.
If \arg{medium} is not grafted to a sheet or the medium's sheet is not currently
mirrored on a display server, \cl{medium-drawable} returns \cl{nil}.

Programmers can use this function to get a host window system object that can be
manipulated using the functions of the host window system.  This might be done
in order to explicitly trade performance against portability.

\Defmethod {port} {(medium \cl{basic-medium})}

If \arg{medium} is both grafted to a sheet and the sheet is currently mirrored
on a display server, this returns the port with which \arg{medium} is
associated.  Otherwise it returns \cl{nil}.

\subsubsection {Grafting and Degrafting of Mediums}

The following generic functions are the protocol-level functions responsible for
the allocating, deallocating, grafting, and degrafting of mediums.  They are not
intended for general use by programmers.

\Defgeneric {allocate-medium} {port sheet}

Allocates a medium from the \term{port} \arg{port}'s medium resource, or calls
\cl{make-medium} on the port to create a new medium if the resource is empty or
the port does not maintain a resource of mediums.  The resulting medium will
have its default characteristics determined by \arg{sheet}.

\Defgeneric {deallocate-medium} {port medium} 

Returns the \term{medium} \arg{medium} to \arg{port}'s medium resource.

\Defgeneric {make-medium} {port sheet}

Creates a new medium for the \term{port} \arg{port}.  The new medium will have
its default characteristics determined by \arg{sheet}.

\Defgeneric {engraft-medium} {medium port sheet}

Grafts the \term{medium} \arg{medium} to the \term{sheet} \arg{sheet} on the
\term{port} \arg{port}.  

The default method on \cl{basic-medium} will set \cl{medium-sheet} on
\arg{medium} to point to the sheet, and will set up the medium state (foreground
ink, background ink, and so forth) from the defaults gotten from {sheet}.  Each
implementation may specialize this generic function in order to set up such
things as per-medium ink caches, and so forth.

\Defgeneric {degraft-medium} {medium port sheet} 

Degrafts the \term{medium} \arg{medium} from the \term{sheet} \arg{sheet} on the
\term{port} \arg{port}.

The default method on \cl{basic-medium} will set \cl{medium-sheet} back to
\cl{nil}.  Each implementation may specialize this generic function in order to
clear any caches it has set up, and so forth.


\section {Repaint Protocol}

The repaint protocol is the mechanism whereby a program keeps the display
up-to-date, reflecting the results of both synchronous and asynchronous events.
The repaint mechanism may be invoked by user programs each time through their
top-level command loop.  It may also be invoked directly or indirectly as a
result of events received from the display server host.  For example, if a
window is on display with another window overlapping it, and the second window
is buried, a ``damage notification'' event may be sent by the server; CLIM would
cause a repaint to be executed for the newly-exposed region.


\subsection {Repaint Protocol Functions}

\Defgeneric {queue-repaint} {sheet repaint-event} 

Requests that the repaint event \arg{repaint-event} be placed in the input queue
of the \term{sheet} \arg{sheet}.  A program that reads events out of the queue
will be expected to call \cl{handle-event} for the sheet using the repaint
region gotten from \arg{repaint-event}.

\Defgeneric {handle-repaint} {sheet region} 

Implements repainting for a given sheet class.  \term{sheet} is the sheet to
repaint and \arg{region} is the region to repaint.

\Defgeneric {repaint-sheet} {sheet region} 

Recursively causes repainting of the \term{sheet} \arg{sheet} and any of its
descendants that overlap the \term{region} \arg{region}.

All CLIM implementations must support repainting for regions that are rectangles
or region sets composed entirely of rectangles.


\subsection {Repaint Protocol Classes}

The following classes implement the standard repaint protocols.  None of the
three following classes are instantiable.

\Defclass {standard-repainting-mixin}

Defines a \cl{dispatch-repaint} method that calls \cl{queue-repaint}.

\Defclass {immediate-repainting-mixin}

Defines a \cl{dispatch-repaint} method that calls \cl{handle-repaint}.

\Defclass {sheet-mute-repainting-mixin}

Defines a \cl{dispatch-repaint} method that calls \cl{queue-repaint}, and a
method on \cl{repaint-sheet} that does nothing.  This means that its children
will be recursively repainted when the repaint event is handled.


\section {Sheet Notification Protocol}

The notification protocol allows sheet clients to be notified when a sheet
hierarchy is changed.  Sheet clients can observe modification events by
providing \cl{:after} methods for functions defined by this protocol.

\subsection {Relationship to Window System Change Notifications}

\issue {RSL} {More to be written.}

\defgeneric {note-sheet-grafted}   {sheet}
\defgeneric {note-sheet-degrafted} {sheet}
\defgeneric {note-sheet-adopted}   {sheet}
\defgeneric {note-sheet-disowned}  {sheet}
\defgeneric {note-sheet-enabled}   {sheet}
\Defgeneric {note-sheet-disabled}  {sheet}

These notification functions are invoked when the state change has been made to
the \term{sheet} \arg{sheet}.


\subsection {Sheet Geometry Notifications}

\issue {RSL} {More to be written.}

\defgeneric {note-sheet-region-changed} {sheet}
\Defgeneric {note-sheet-transformation-changed} {sheet}

These notification functions are invoked when the region or transformation of
the \term{sheet} \arg{sheet} has been changed.  When the regions and
transformations of a sheet are changed directly, the client is required to call
\cl{note-sheet-region-changed} or \cl{note-sheet-transformation-changed}.


\chapter {Ports, Grafts, and Mirrored Sheets}
\label{ports-and-grafts}

\section {Introduction}

A sheet hierarchy must be attached to a display server so as to permit input and
output.  This is managed by the use of \concept{ports} and \concept{grafts}.

\section {Ports}

A \concept{port} is a logical connection to a display server.  It is responsible
for managing display output and server resources, and for handling incoming
input events.  Typically, the programmer will create a single port that will
manage all of the windows on the display.

A port is described with a \concept{server path}.  A server path is a list whose
first element is a keyword that selects the kind of port.  The remainder of the
server path is a list of alternating keywords and values whose interpretation is
port type-specific.

\Defprotoclass {port}

The protocol class that corresponds to a port.
\IfYouWantClass {a} {port} {port}

All of the subclasses of \cl{port} are mutable.

\Defpredicate {portp} {object}

Returns \term{true} if \arg{object} is a \term{port}, otherwise returns
\term{false}.

\Defclass {basic-port}

The basic class on which all CLIM ports are built, a subclass of \cl{port}.
\AbstractClass


\Defun {find-port} {\rest initargs \key (server-path \cl{*default-server-path*}) \allow}

Finds a port that provides a connection to the window server addressed by
\arg{server-path}.  If no such connection exists, a new connection will be
constructed and returned.  The initargs in \arg{initargs} will be passed to the
function that constructed the new port.

\Defvar {*default-server-path*}

This special variable is used by \cl{find-port} and its callers to default the
choice of a display service to locate.  Binding this variable in a dynamic
context will affect the defaulting of this argument to these functions.  This
variable will be defaulted according to the environment.  In the Unix
environment, for example, CLIM will attempt to set this variable based on the
value of the \cl{DISPLAY} environment variable.

The value of \cl{*default-server-path*} is a cons of a port type followed by a
list of initargs.

The following are the recommendations for port types and their initargs.  This
list is not intended to be comprehensive, nor is it required that a CLIM
implementation support any of these port types.

\def\Defdp #1 #2 {\Dodocf {#1} {#2} {Server~Path}}

\Defdp {:clx} {\key host display-id screen-id}

Given this server path, \cl{find-port} finds a port for the X server on the
given \arg{host}, using the \arg{display-id} and \arg{screen-id}.

On a Unix host, if these values are not supplied, the defaults come from the
\cl{DISPLAY} environment variable.  Each CLIM implementation must describe how
it uses such environment variables.

\Defdp {:motif} {\key host display-id screen-id}

Given this server path, \cl{find-port} finds a port for a Motif X server on the
given \arg{host}, using the \arg{display-id} and \arg{screen-id}.

On a Unix host, if these values are not supplied, the defaults come from the
\cl{DISPLAY} environment variable.

\Defdp {:openlook} {\key host display-id screen-id}

Given this server path, \cl{find-port} finds a port for an OpenLook X server on
the given \arg{host}, using the \arg{display-id} and \arg{screen-id}.

On a Unix host, if these values are not supplied, the defaults come from the
\cl{DISPLAY} environment variable.

\Defdp {:genera} {\key screen}

Given this server path, \cl{find-port} finds a port for local Genera platform on
the screen object \arg{screen}.  \arg{screen} defaults to \arg{tv:main-screen},
but could also be an object return from \cl{color:find-color-screen}.


\Defgeneric {port} {object}

Returns the port associated with \arg{object}.  \cl{port} is defined for all
sheet classes (including grafts and streams that support the CLIM graphics
protocol), mediums, and application frames.  For degrafted sheets or other
objects that aren't currently associated with particular ports, \cl{port}
will return \cl{nil}.

\Defmacro {with-port-locked} {(port) \body body} 

Executes \arg{body} after grabbing a lock associated with the \term{port}
\arg{port}, which may be a port or any object on which the function \cl{port}
works.  If \arg{object} currently has no port, \arg{body} will be executed
without locking.

\arg{body} may have zero or more declarations as its first forms.


\Defun {map-over-ports} {function}

Invokes \arg{function} on each existing port.  Function is a function of one
argument, the port; it has dynamic extent.

\Defgeneric {port-server-path} {port}

Returns the server path associated with the \term{port} \arg{port}.

\Defgeneric {port-name} {port}

Returns an implementation-dependent string that is the name of the port.  For
example, a \cl{:clx} port might have a name of \cl{"summer:0.0"}.

\Defgeneric {port-type} {port}

Returns the type of the port, that is, the first element of the server path
spec.


\defgeneric {port-properties} {port indicator}
\Defgeneric {(setf port-properties)} {property port indicator}

These functions provide a port-based property list.  They are primarily intended
to support users of CLIM that may need to associate certain information with
ports.  For example, the implementor of a special graphics package may need to
maintain resource tables for each port on which it is used.

\Defgeneric {restart-port} {port}

In a multi-process Lisp, \cl{restart-port} restarts the global input processing
loop associated with the \term{port} \arg{port}.  All pending input events are
discarded.  Server resources may or may not be released and reallocated during
or after this action.

\Defgeneric {destroy-port} {port}

Destroys the connection with the window server represented by the \term{port}
\arg{port}.  All sheet hierarchies that are associated with \arg{port} are
forcibly degrafted by disowning the children of grafts on \arg{port} using
\arg{sheet-disown-child}.  All server resources utilized by such hierarchies or
by any graphics objects on \arg{port} are released as part of the connection
shutdown.


\section {Grafts}

A \concept{graft} is a special sheet that is directly connected to a display
server.  Typically, a graft is the CLIM sheet that represents the root window of
the display.  There may be several grafts that are all attached to the same root
window; these grafts may have differing coordinate systems.

To display a sheet on a display, it must have a graft for an ancestor.  In
addition, the sheet and all of its ancestors must be enabled, including the
graft.  In general, a sheet becomes grafted when it (or one of its ancestors) is
adopted by a graft.

\Defgeneric {sheet-grafted-p} {sheet}

Returns \term{true} if any of the sheet's ancestors is a graft, otherwise
returns \term{false}.


\Defun {find-graft} {\key (server-path \cl{*default-server-path*})
                          (port \cl{(find-port :server-path server-path)})
                          (orientation \cl{:default}) (units \cl{:device})} 

Finds a graft that represents the display device on the port \arg{port} that
also matches the other supplied parameters.  If no such graft exists, a new
graft is constructed and returned.

If \arg{server-path} is supplied, \cl{find-graft} finds a graft whose port
provides a connection to the window server addressed by \arg{server-path}.

It is an error to provide both \arg{port} and \arg{server-path} in a call to
\cl{find-graft}.

\arg{orientation} specifies the orientation of the graft's coordinate
system.  Supported values are \cl{:default} and \cl{:graphics}, which have the
meanings describe below:

\begin{itemize}
\item \cl{:default}---a coordinate system with its origin is in the upper left
hand corner of the display device with $y$ increasing from top to bottom and $x$
increasing from left to right.

\item \cl{:graphics}---a coordinate system with its origin in the lower left
hand corner of the display device with $y$ increasing from bottom to top and $x$
increasing from left to right.
\end{itemize}

\arg{units} specifies the units of the coordinate system and defaults to
\cl{:device}, which means the device units of the host window system (such as
pixels).  Other supported values include \cl{:inches}, \cl{:millimeters}, and
\cl{:screen-sized}, which means that one unit in each direction is the width and
height of the display device.

\issue {RSL} {I don't know how much of this is obsolete.}

\Defgeneric {graft} {object}

Returns the graft currently associated with \arg{object}.  \cl{graft} is
defined for all sheet classes (including streams that support the CLIM
graphics protocol), mediums, and application frames.  For degrafted sheets or
other objects that aren't currently associated with a particular graft,
\cl{graft} will return \cl{nil}.

\Defun {map-over-grafts} {function port}

Invokes \arg{function} on each existing graft associated with the \term{port}
\arg{port}.  \arg{function} is a function of one argument, the graft; it has
dynamic extent.

\Defmacro {with-graft-locked} {(graft) \body body} 

Executes \arg{body} after grabbing a lock associated with the \term{graft}
\arg{graft}, which may be a graft or any object on which the function
\cl{graft} works.  If \arg{object} currently has no graft, \arg{body} will
be executed without locking.

\arg{body} may have zero or more declarations as its first forms.


\Defgeneric {graft-orientation} {graft}

Returns the orientation of the \term{graft} \arg{graft}'s coordinate system.
The returned value will be either \cl{:default} or \cl{:graphics}.  The meanings
of these values are the same as described for the orientation argument to
\cl{find-graft}.

\Defgeneric {graft-units} {graft}

Returns the units of the \term{graft} \arg{graft}'s coordinate system.  The
returned value will be one of \cl{:device}, \cl{:inches}, \cl{:millimeters}, or
\cl{:screen-sized}.  The meanings of these values are the same as described for
the units argument to \cl{find-graft}.

\defgeneric {graft-width} {graft \key (units \cl{:device})}
\Defgeneric {graft-height} {graft \key (units \cl{:device})}

Returns the width and height of the \term{graft} \arg{graft} (and by extension
the associated host window) in the units indicated.  \arg{Units} may be any of
\cl{:device}, \cl{:inches}, \cl{:millimeters}, or \cl{:screen-sized}.  The
meanings of these values are the same as described for the units argument to
\cl{find-graft}.  Note that if a \arg{unit} of \cl{:screen-sized} is specified, both
functions will return a value of \cl{1}.

\defun {graft-pixels-per-millimeter} {graft}
\Defun {graft-pixels-per-inch} {graft}

Returns the number of pixels per millimeter or inch of the \term{graft}
\arg{graft}.  These functions are provided as a convenience to programmers and
can be easily written in terms of \cl{graft-width} or \cl{graft-height}.

\issue {Rao} {Do we want to support non-square pixels?  If so, these functions
aren't sufficient.}


\section {Mirrors and Mirrored Sheets}

A \concept{mirrored sheet} is a special class of sheet that is attached directly
to a window on a display server.  Grafts, for example, are always mirrored
sheets.  However, any sheet anywhere in a sheet hierarchy may be a mirrored
sheet.  A mirrored sheet will usually contain a reference to a window system
object, called a mirror.  For example, a mirrored sheet attached to an X11
server might have an X window system object stored in one of its slots.
Allowing mirrored sheets at any point in the hierarchy enables the adaptive
toolkit facilities.

Since not all sheets in the hierarchy have mirrors, there is no direct
correspondence between the sheet hierarchy and the mirror hierarchy.  However,
on those display servers that support hierarchical windows, the hierarchies must
be parallel.  If a mirrored sheet is an ancestor of another mirrored sheet,
their corresponding mirrors must have a similar ancestor/descendant
relationship.

CLIM interacts with mirrors when it must display output or process events.
On output, the mirrored sheet closest in ancestry to the sheet on which we wish
to draw provides the mirror on which to draw.  The mirror's drawing clipping
region is set up to be the intersection of the user's clipping region and the
sheet's region (both transformed to the appropriate coordinate system) for the
duration of the output.  On input, events are delivered from mirrors to the
sheet hierarchy.  The CLIM port must determine which sheet shall receive events
based on information such as the location of the pointer.

In both of these cases, we must have a coordinate transformation that converts
coordinates in the mirror (so-called ``native'' coordinates) into coordinates
in the sheet and vice-versa.

\Defclass {mirrored-sheet-mixin}

This class is mixed in to sheet classes that can be directly mirrored.


\subsection {Mirror Functions}

\issue {SWM} {What kind of an object is a mirror?  Is it the Lisp object that is
the handle to the actual toolkit window or gadget?}

\Defgeneric {sheet-direct-mirror} {sheet}

Returns the mirror of the \term{sheet} \arg{sheet}.  If the sheet is not a
subclass of \cl{mirrored-sheet-mixin}, this will return \cl{nil}.  If the sheet
is a subclass of \cl{mirrored-sheet-mixin} and does not currently have a mirror,
\cl{sheet-direct-mirror} will return \cl{nil}.

\Defgeneric {sheet-mirrored-ancestor} {sheet}

Returns the nearest mirrored ancestor of the \term{sheet} \arg{sheet}.

\Defgeneric {sheet-mirror} {sheet}

Returns the mirror of the \term{sheet} \arg{sheet}.  If the sheet is not itself
mirrored, \cl{sheet-mirror} returns the direct mirror of its nearest mirrored
ancestor.  \cl{sheet-mirror} could be implemented as:

\begin{verbatim}
(defun sheet-mirror (sheet)
  (sheet-direct-mirror (sheet-mirrored-ancestor sheet)))
\end{verbatim}

\Defgeneric {realize-mirror} {port mirrored-sheet}

Creates a mirror for the \term{sheet} \arg{mirrored-sheet} on the \term{port}
\arg{port}, if it does not already have one.  The returned value is the sheet's
mirror; the type of this object is implementation dependent.

\Defgeneric {destroy-mirror} {port mirrored-sheet}

Destroys the mirror for the \term{sheet} \arg{mirrored-sheet} on the \term{port}
\arg{port}.

\Defgeneric {raise-mirror} {port sheet}

Raises the \term{sheet} \arg{sheet}'s mirror to the top of all of the host
windows on the \term{port} \arg{port}.  \arg{sheet} need not be a directly
mirrored sheet.

\Defgeneric {bury-mirror} {port sheet}

Buries the \term{sheet} \arg{sheet}'s mirror at the bottom of all of the host
windows on the \term{port} \arg{port}.  \arg{sheet} need not be a directly
mirrored sheet.

\Defmethod {port} {(sheet \cl{basic-sheet})}

If \arg{sheet} is currently mirrored on a display server, this returns the port
with which \arg{sheet} is associated.  Otherwise it returns \cl{nil}.


\subsection {Internal Interfaces for Native Coordinates}

\issue {SWM} {Do these functions work on any sheet, or only on sheets that have
a mirror, or only on sheets that have a direct mirror?  Also, define what
``native coordinates'' are.  Also, do \cl{sheet-device-transformation} and
\cl{sheet-device-region} really account for the user's transformation and
clipping region?}

\Defgeneric {sheet-native-transformation} {sheet}

Returns the transformation for the \term{sheet} \arg{sheet} that converts sheet
coordinates into native coordinates.  The object returned by this function is
volatile, so programmers must not depend on the components of the object
remaining constant.

\Defgeneric {sheet-native-region} {sheet}

Returns the region for the \term{sheet} \arg{sheet} in native coordinates.  The
object returned by this function is volatile, so programmers must not depend on
the components of the object remaining constant.

\Defgeneric {sheet-device-transformation} {sheet}

Returns the transformation used by the graphics output routines when drawing on
the mirror.  This is the composition of the sheet's native transformation and
the user transformation.  The object returned by this function is volatile, so
programmers must not depend on the components of the object remaining constant.

\Defgeneric {sheet-device-region} {sheet}

Returns the actual clipping region to be used when drawing on the mirror.  This
is the intersection of the user's clipping region (transformed by the device
transformation) with the sheet's native region.  The object returned by this
function is volatile, so programmers must not depend on the components of the
object remaining constant.

\Defgeneric {invalidate-cached-transformations} {sheet}

\cl{sheet-native-transformation} and \cl{sheet-device-transformation} typically
cache the transformations for performance reasons.
\cl{invalidate-cached-transformations} clears the cached native and device
values for the \term{sheet} \arg{sheet}'s transformation and clipping region.
It is invoked when a sheet's native transformation changes, which happens when a
sheet's transformation is changed or when \cl{invalidate-cached-transformations}
is called on any of its ancestors.

\Defgeneric {invalidate-cached-regions} {sheet}

\cl{sheet-native-region} and \cl{sheet-device-region} typically cache the
regions for performance reasons.  \cl{invalidate-cached-regions} clears the
cached native and device values for the \term{sheet} \arg{sheet}'s native
clipping region.  It is invoked when a sheet's native clipping region changes,
which happens when the clipping region changes or when
\cl{invalidate-cached-regions} is called on any of its ancestors.


\comment {Here is stuff from the old Silica chapters that should perhaps
be incorporated into the above:

\section {Dynamic Output Contract}

The \cl{dynamic-output-contract} class allows for the dynamic selection and use
of properly incorporated graphics package.  Graphics packages are incorporated
into the Silica framework (provided to clients of dynamic output) by
implementing a \cl{display medium} class.  A display medium combines state about
the target sheet, imaging parameters and options supported by the medium, and
port specific implementation data.  The implementation of a display medium type
is discussed in Section~\ref{display-medium}.

A client can get a display medium for a dynamic output sheet in one of two ways:

\begin{enumerate}
\item A medium may be allocated to the sheet by passing in the \cl{:medium-type}
initarg (for example, \cl{:clim} gets a CLIM graphics medium) at sheet creation.

\item A programmer can use \cl{using-display-medium} to temporary allocate a
medium that uses the sheet as its target.
\end{enumerate}

\Definitarg {:medium-type}

This initarg can be used with dynamic output contract.  If this initarg is used,
than a medium will be allocated for the sheet when it is grafted.  \cl{:clim} is
the only value guaranteed to be implemented across all ports.

\section {Providing a Display Medium Type}

The graphics package independence of \cl{standard-output-contract} is based on
the concept of display media.  Display media provide a context that allows the
window system to control output performed by clients.  In addition, display
media may provide auxiliary state for managing client imaging parameters.
Graphics packages are incorporated into Silica (that is, made usable by users of
standard output) by implementing a \concept{display medium} class.

A medium is an implementation or incorporation of a graphics package into the
Silica framework.  This section defines the protocols for implementing a medium
type and in particular a display medium type.

\Defclass {display-medium}

Display mediums are subclasses of standard output that incorporate a graphics
package into the Silica framework.
  
A display medium class must implement two protocol: a client interface to the
graphics package (which is completely defined by the graphics package being
incorporated), and an interface required by Silica.  The client interface is
generally the same as the client interface to the graphics package (that's the
point of providing that graphics package) though slight additions may be made to
improve performance.  The \cl{standard-output-contract} requirements are
sufficient to allow using the graphics package on a sheet.  They consist of the
following areas:

\begin{itemize}
\item {\bf View Control}.  Generic functions for setting the clipping region and
output transformation as required by windowing policies.  These routines must
not interfere with the use of clipping and transformations by the client, if the
client interface allows these things.

\item {\bf Output Protection}.  This includes validating that proper clipping
regions and transformation are being used, and locking out changes that might
influence these.  Protection granularity depends significantly on the graphics
package and on what kind of client interface to graphics routines are provided.

\item {\bf Integration of Types}. The region and transformation types used by
the graphics package should be integrated into the kernel or coerced at
appropriate interfaces, so that clients don't have to use different kinds of
regions to talk to windowing functionality and output functionality.
\end{itemize}

\defgeneric {(setf silica-clipping-region)} {display-medium}
\Defgeneric {(setf silica-medium-transformation)} {display-medium}

A display medium must clip and transform all output according to the values
returned by these functions.  A display medium class can provide methods on the
setf generic functions to update state or to mark the need to update state or
whatever.

\defgeneric {device-clipping-region} {display-medium} 
\Defgeneric {device-transformation} {display-medium}

\Defmacro {with-output-protection} {(dm) \body body}

All graphics routines provided by the display medium must insure that imaging
happens with a protected context using this macro.  This allows the window
system to insure that the display medium clipping regions and transformations
will not changed during the scope of the imaging operation.

\chapter {Input Processing}

Silica provides an input architecture and standard functionality for notifying
clients of input that is distributed to their sheets.  Input includes the
pointer entering and exiting sheets, pointer motion as allowed by performance,
and mouse button and keyboard events.

Programmers who would like to design specialized input machinery need to
understand the details of Silica's input distribution architecture.  This
section covers that topic.  Most users of higher levels of CLIM need not
concern themselves with the details presented here.

Silica separates device input processing into two phases:  distribution
(determining which sheet should receive the input), and delivery (translating
the raw input into a representation suitable to the client of the recipient
sheet).  The design goals for the input architecture include allowing flexible
selection of both distribution and delivery policies as well as providing
standard policies that are portable across a wide variety of host window servers
and sufficient for most applications.  This input architecture supports both
event-oriented and stream-oriented input interfaces; and both single and
multiple process environments.

The flow of input processing proceeds as follows.  Device events come in from a
port in a window server specific way.  Each device event is directed to an event
distributor that is responsible for determining the recipient of the input, and
possibly also for generating crossing events on mouse motion.  A standard
distributor is provided and chosen by default for most ports.

After a recipient is found for a device event, the device event is passed to the
input contract of the sheet.  The input contract translates and delivers the
event to the sheet's client.  This means that an input contract must define a
method for converting and delivery the device event of each port type that it
supports.  This approach allows delayed representation of input based on the
final recipient of the input.  The standard input contracts are described below.

Each port has a event processor that processes all of the
implementation-specific device events.  This port event processor may run in a
separate per-port process or can be invoked directly by higher levels in a
single-process environment.  For example, in a multiprocess environment a port
event handler would run code similar to the following:

\begin{verbatim}
   (loop (process-next-event display))
\end{verbatim}

\Defgeneric {process-next-event} {port \key timeout}

This function provides a standard interface for one pass through a port's event
processing loop.  It returns \term{true} if an event was processed, otherwise it
returns \term{false} (meaning that the function timed out).  A port
implementation must provide a method for this function that reads the next
window server specific device event (blocking if necessary) and then invokes the
port's event distributor via a call to \cl{distribute-device-event}.  This
method is required to pass along some necessary keyword arguments along with any
port specific keyword args that will eventually be parsed by port-specific
methods defined on the input contract of the recipient sheet.

\Defgeneric {distribute-device-event} {distributor port source \key \allow}

A distributor implementation must provide a method for this function.
\arg{Source} indicates the sheet that corresponds to the host window that the
event was delivered to by the window server.  This method is responsible for
invoking \cl{dispatch-device-event} or \cl{dispatch-crossing-event} on input
contract of sheet's.

\Defclass {standard-event-distributor} 

This class implements the standard distribution policy supported by Silica.
Pointer button and motion events are distributed to deepest sheet in the
hierarchy that contains the pointer at the time that the event is generated.
Keyboard events can be distributed either to a particular sheet or to the same
sheet that would receive pointer events.

\defgeneric {dispatch-device-event} {port input-contract \key sheet \allow}
\Defgeneric {dispatch-crossing-event} {port input-contract \key sheet \allow}

These functions are invoked by an event distributor.  Typically the
input-contract is the same object as the sheet and some generic as well as some
port-specific keywords arguments indicating the device event are passed in.
Input contract implementations provide methods specialized on each of the ports
that the input contract is supported on.  These methods are responsible for
parsing the keyword arguments and delivering the event to the client of the
sheet.


%\Defgeneric {port-event-distributor} {port}

%\Defgeneric {port-cursor} {port}

\section {Input Contracts}

Standard event delivery is implemented by the \cl{standard-input-contract}
class.  This input contract type provides methods for the appropriate generic
functions of the Event Distribution protocol that invoke \cl{queue-event} on the
recipient sheet and the event.  Clients can pass in an event queue when a
standard input contract is initialized or change the queue associated with the
sheet later.

\keepout{This input contract type provides event delivery methods that converts
the port-specific event keywords into the standard event representation and
then invokes \cl{queue-event} on the recipient sheet and the event.}

\Defclass {standard-input-contract}

In addition to the standard input contract, two other contracts are provided:
\cl{mute-input-contract} and \cl{invoking-input-contract}.  In addition, the
\cl{bursting-input-queuer} mixin is provided as a convenient addition to the
protocol implemented by \cl{invoking-input-contract}.

\Defclass {mute-input-contract}

This class supports sheets that do not wish to receive input.

\Defclass {invoking-input-contract}

The \cl{invoking-input-contract} is different from the standard input contract
in at least two significant ways.  First, it converts the host-specific event
into a set of keyword arguments and invokes \cl{handle-input} on the recipient
sheet and these arguments (as opposed to queuing a structured event).  And
second, the client is expected to do its input handling synchronously within the
port's event handling process.  (In single process environments, the difference
is that the input is done during queue-input, rather than at a some other place
in the application input loop).
        
\Defgeneric {queue-input} {client \key type sheet state x y time \allow} 

This function is invoked by the invoking input contract.  \arg{client} is
typically the recipient sheet itself, since the creator of the sheet will have
additional super classes added to the sheet's sheet class.

\Defclass {bursting-input-queuer}

This class provides standard methods for \cl{queue-input}.  The method for
\cl{queue-input} provided by input handler examines the \arg{type} and
dispatches to one of the event handling function described below.

\Defgeneric {enter-region} {client \key \allow}
\Defgeneric {exit-region} {client \key \allow}

These generic functions are invoked by \cl{deliver-event} when the cursor enters or
exits an active region

\issue {SWM} {What is \cl{deliver-event}?  Why do these allow other keywords?}

\chapter {Other Input Facilities}

\section {Standard Polling Interfaces}

\Defgeneric {poll-pointer} {sheet}

This function provides polling support.  Returns X, Y, and state.  This function
can be used to obtain the location of the pointer device and the state of mouse
buttons and modifier keys.

} %% end of long comment
