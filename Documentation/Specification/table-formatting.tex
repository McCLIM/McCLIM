% -*- Mode: LaTeX; Package: CLIM-USER -*-

\chapter {Table Formatting}
\label {table-formatting}

CLIM provides a mechanism for tabular formatting of arbitrary output.

To employ these facilities the programmer annotates some output-generating code
with advisory macros that describe the high-level formatting constraints, for
example, what parts of code produce a row of the table, what parts of that
produce the cells in the row.

For example, the following produces a table consisting of three columns
containing a number, its square, and its cube.  The output can be seen in
Figure~\ref{table-example}.

\begin{verbatim}
(defun table-test (count stream)
  (fresh-line stream)
  (formatting-table (stream :x-spacing '(3 :character))
    (dotimes (i count)
      (formatting-row (stream)
        (formatting-cell (stream :align-x :right)
          (prin1 i stream))
        (formatting-cell (stream :align-x :right)
          (prin1 (* i i) stream))
        (formatting-cell (stream :align-x :right)
          (prin1 (* i i i) stream))))))
\end{verbatim}

\begin{figure}
\centerline{\includegraphics{table-example}}
\caption{\label{table-example} Example of tabular output.}
\end{figure}

The general contract of these facilities is described in the next section.


\section {Overview of Table Formatting Facilities}

In general, table formatting involves a sharing of responsibilities between
user-written code and CLIM code.  Code that employs only the lower level output
facilities has full control over ``where every piece of ink goes'' in the
output.  In contrast, code that employs CLIM's table formatting facilities
passes control to CLIM at a higher level.  The programmer benefits by being able
to specify the appearance of output in more compact abstract terms, and by not
having to write the code that constrains the output to appear in proper tabular
form.

Tabular output consists of a rectangular array of pieces of output corresponding
to the bounding rectangles of the output.  Each piece of output forms the
contents of a \concept{table cell}.  There is no restriction on the contents of
a table cell; cells may contain text, graphics, even other tables.  For purposes
of this discussion, we draw a strong distinction between specifying what goes in
a cell, and specifying how the cells are arranged to form a table.

Specifying the contents of a cell is the responsibility of the programmer.  A
programmer using the table formatting facilities can predict the appearance of
any individual cell by simply looking at the code for that cell.  A cell's
appearance does not depend upon where in the table it lies, for instance.  The
only thing about a cell's appearance that cannot be predicted from that cell
alone is the amount of space the table formatting has to introduce in order to
perform the desired alignment.

Specifying the relative arrangements of cells to form a table is the
responsibility of CLIM based on the advice of the programmer.  The programmer
advises CLIM about extra space to put between rows or columns, for instance, but
does not directly control the absolute positioning of a cell's contents.

For purposes of understanding table formatting, the following model may
be used. 
\begin{itemize}
\item The code for a cell draws to a stream that has a ``private''  (local to
that cell) drawing plane.

\item After output for a cell has finished, the bounding rectangle of all output
on the ``private'' drawing plane is found.  The region within that bounding
rectangle forms the contents of a cell.

\item Additional rectangular regions, containing only background ink, are
attached to the edges of the cell contents.  These regions ensure that the
cells satisfy the tabular constraints that within a row all cells have the same
height, and within a column all cells have the same width.  CLIM may also
introduce additional background for other purposes as described below.

\item The cells are assembled into rows and columns.
\end{itemize}

Some tables are ``multiple column'' tables, in which two or more rows of the
table are placed side by side (usually with intervening spacing) rather than
all rows being aligned vertically.  Multiple column tables are generally used
to produce a table that is more esthetically pleasing, or to make more
efficient use of space on the output device.  When a table is a multiple
column table, one additional step takes place in the formatting of the table:
the rows of the table are rearranged into multiple columns in which some rows
are placed side by side.

The advice that the programmer gives to CLIM on how to assemble the table
consists of the following:
\begin{itemize}
\item How to place the contents of the cell within the cell (such as centered
vertically, flush-left, and so forth). The possibilities for this advice are
described below.

\item Optionally, how much additional space to insert between columns and
between rows of the table.

\item Optionally, whether to make all columns the same size.
\end{itemize}

The advice describing how to place the contents of the cell within the cell
consists of two pieces---how to constrain the cell contents in the horizontal
direction, and how to constrain them in the vertical direction.

\section {Table Formatting Functions}

\Defmacro {formatting-table} {(\optional stream
                               \key x-spacing y-spacing
                                    multiple-columns multiple-columns-x-spacing
                                    equalize-column-widths 
                                    (move-cursor \cl{t}) record-type \allow)
                              \body body}

Binds the local environment in such a way that the output of \arg{body} will be done
in a tabular format.  This must be used in conjunction with \cl{formatting-row}
or \cl{formatting-column}, and \cl{formatting-cell}.  The table is placed so
that its upper left corner is at the current text cursor position of
\arg{stream}.  If the boolean \arg{move-cursor} is \term{true} (the default),
then the text cursor will be moved so that it immediately follows the last cell
of the table.

The returned value is the output record corresponding to the table.

\arg{stream} is an output recording stream to which output will be done.  The
\arg{stream} argument is not evaluated, and must be a symbol that is bound to a
stream.  If \arg{stream} is \cl{t} (the default), \cl{*standard-output*} is
used.  \arg{body} may have zero or more declarations as its first forms.

\arg{x-spacing} specifies the number of units of spacing to be inserted between
columns of the table; the default is the width of a space character in the
current text style.  \arg{y-spacing} specifies the number of units of spacing to
be inserted between rows in the table; the default is the default vertical
spacing of the stream.  Possible values for these two options option are:

\begin{itemize}
\item An integer---a size in the current units to be used for spacing.

\item A string or character---the spacing is the width or height of the string
or character in the current text style. 

\item A function---the spacing is the amount of horizontal or vertical space the
function would consume when called on the stream. 

\item A list---the list is of the form \cl{(\arg{number} \arg{unit})}, where
\arg{unit} is one of \cl{:point}, \cl{:pixel}, \cl{:mm}, \cl{:character}, or
\cl{:line}.  When \arg{unit} is \cl{:character}, the width of an ``M'' in the
current text style is used as the width of one character.
\end{itemize}

\arg{multiple-columns} is either \cl{nil}, \cl{t}, or an integer.  If it is
\cl{t} or an integer, the table rows will be broken up into a multiple columns.
If it is \cl{t}, CLIM will determine the optimal number of columns.  If it is an
integer, it will be interpreted as the desired number of columns.
\arg{multiple-columns-x-spacing} has the same format as \arg{x-spacing}.  It
controls the spacing between the multiple columns.  It defaults to the value of
the \arg{x-spacing} option.

When the boolean \arg{equalize-column-widths} is \term{true}, CLIM will make all
of the columns have the same width (the width of the widest cell in any column
in the entire table).

\arg{record-type} specifies the class of output record to create.  The default
is \cl{standard-table-output-record}.  This argument should only be supplied by
a programmer if there is a new class of output record that supports the table
formatting protocol.


\Defmacro {formatting-row} {(\optional stream \key record-type \allow)
                            \body body} 

Binds the local environment in such a way the output of \arg{body} will be
grouped into a table row.  All of the output performed by \arg{body} becomes the
contents of one row.  This must be used inside of \cl{formatting-table}, and in
conjunction with \cl{formatting-cell}.

\arg{stream} is an output recording stream to which output will be done.  The
\arg{stream} argument is not evaluated, and must be a symbol that is bound to a
stream.  If \arg{stream} is \cl{t} (the default), \cl{*standard-output*} is
used.  \arg{body} may have zero or more declarations as its first forms.

Once a table has had a row added to it via \cl{formatting-row}, no columns may
be added to it.

\arg{record-type} specifies the class of output record to create.  The default
is \cl{standard-row-output-record}.  This argument should only be supplied by a
programmer if there is a new class of output record that supports the row
formatting protocol.


\Defmacro {formatting-column} {(\optional stream \key record-type \allow)
                               \body body} 

Binds the local environment in such a way the output of \arg{body} will be
grouped into a table column.  All of the output performed by \arg{body} becomes
the contents of one column.  This must be used inside of \cl{formatting-table},
and in conjunction with \cl{formatting-cell}.

\arg{stream} is an output recording stream to which output will be done.  The
\arg{stream} argument is not evaluated, and must be a symbol that is bound to a
stream.  If \arg{stream} is \cl{t} (the default), \cl{*standard-output*} is
used.  \arg{body} may have zero or more declarations as its first forms.

Once a table has had a column added to it via \cl{formatting-column}, no rows
may be added to it.

\arg{record-type} specifies the class of output record to create.  The default
is \cl{standard-column-output-record}.  This argument should only be supplied
by a programmer if there is a new class of output record that supports the
column formatting protocol.


\Defmacro {formatting-cell} {(\optional stream
                              \key (align-x \cl{':left}) (align-y \cl{':baseline}) 
                                   min-width min-height record-type \allow)
                             \body body}

Controls the output of a single cell inside a table row or column, or of a
single item inside \cl{formatting-item-list}.  All of the output performed by
\arg{body} becomes the contents of the cell.

\arg{stream} is an output recording stream to which output will be done.  The
\arg{stream} argument is not evaluated, and must be a symbol that is bound to a
stream.  If \arg{stream} is \cl{t} (the default), \cl{*standard-output*} is
used.  \arg{body} may have zero or more declarations as its first forms.

\arg{align-x} specifies how the output in a cell will be aligned relative to
other cells in the same table column.  The default, \cl{:left}, causes the cells
to be flush-left in the column.  The other possible values are \cl{:right}
(meaning flush-right in the column) and \cl{:center} (meaning centered in the
column).  Each cell within a column may have a different alignment; thus it is
possible, for example, to have centered legends over flush-right numeric data.

\arg{align-y} specifies how the output in a cell will be aligned vertically.
The default, \cl{:baseline}, causes textual cells to be aligned along their
baselines and graphical cells to be aligned at the bottom.  The other possible
values are \cl{:bottom} (align at the bottom of the output), \cl{:top} (align at
the top of the output), and \cl{:center} (center the output in the cell).

\arg{min-width} and \arg{min-height} are used to specify minimum width or height
of the cell.  The default, \cl{nil}, causes the cell to be only as wide or high
as is necessary to contain the cell's contents.  Otherwise, \arg{min-width} and
\arg{min-height} are specified in the same way as the \cl{:x-spacing} and
\cl{:y-spacing} arguments to \cl{formatting-table}.

\arg{record-type} specifies the class of output record to create.  The default
is \cl{standard-cell-output-record}.  This argument should only be supplied by
a programmer if there is a new class of output record that supports the cell
formatting protocol.


\Defmacro {formatting-item-list} {(\optional stream
                                   \key x-spacing y-spacing
                                        n-columns n-rows 
                                        stream-width stream-height max-width max-height
                                        initial-spacing (row-wise \cl{t})
                                        (move-cursor \cl{t}) record-type \allow) 
                                  \body body}

Binds the local environment in such a way that the output of \arg{body} will be
done in an item list (that is, menu) format.  This must be used in conjunction
with \cl{formatting-cell}, which delimits each item.  The item list is placed so
that its upper left corner is at the current text cursor position of
\arg{stream}.  If the boolean \arg{move-cursor} is \term{true} (the default),
then the text cursor will be moved so that it immediately follows the last cell
of the item list.

``Item list output'' is more strictly defined as: each row of the item list
consists of a single cell.  Rows are placed with the first row on top, and each
succeeding row has its top aligned with the bottom of the previous row (plus the
specified \arg{y-spacing}).  Multiple rows and columns are constructed after
laying the item list out in a single column.  Item list output takes place in a
normalized +$y$-downward coordinate system.

If \arg{row-wise} is \term{true} (the default) and the item list requires
multiple columns, each successive element in the item list is layed out from
left to right.  If \arg{row-wise} is \term{false} and the item list requires
multiple columns, each successive element in the item list is layed out below
its predecessor, like in a telephone book.

The returned value is the output record corresponding to the table.

\arg{stream} is an output recording stream to which output will be done.  The
\arg{stream} argument is not evaluated, and must be a symbol that is bound to a
stream.  If \arg{stream} is \cl{t} (the default), \cl{*standard-output*} is
used.  \arg{body} may have zero or more declarations as its first forms.

\arg{x-spacing} specifies the number of units of spacing to be inserted between
columns of the item list; the default is the width of a \verb+#\Space+ character
in the current text style.  \arg{y-spacing} specifies the number of units of
spacing to be inserted between rows in the item list; the default is default
vertical spacing of the stream.  The format of these arguments is as for
\cl{formatting-table}.

When the boolean \arg{equalize-column-widths} is \term{true}, CLIM will make all
of the columns have the same width (the width of the widest cell in any column
in the entire item list).

\arg{n-columns} and \arg{n-rows} specify the number of columns or rows in the
item list.  The default for both is \cl{nil}, which causes CLIM to pick an
aesthetically pleasing layout, possibly constrained by the other options.  If
both \arg{n-columns} and \arg{n-rows} are supplied and the item list contains
more elements than will fit according to the specification, CLIM will format the
item list as if \arg{n-rows} were supplied as \cl{nil}.

\arg{max-width} and \arg{max-height} constrain the layout of the item list.
CLIM will not make the item list any wider than \arg{max-width}, unless it is
overridden by \arg{n-rows}.  It will not make the item list any taller than
\arg{max-height}, unless it is overridden by \arg{n-columns}.

\cl{formatting-item-list} normally spaces items across the entire width of the
stream.  When \arg{initial-spacing} is \term{true}, it inserts some whitespace
(about half as much space as is between each item) before the first item on each
line.  When it is \term{false} (the default), the initial whitespace is not
inserted.

\arg{record-type} specifies the class of output record to create.  The default
is \cl{standard-item-list-output-record}.  This argument should only be
supplied by a programmer if there is a new class of output record that supports
the item list formatting protocol.


\Defun {format-items} {items \key stream printer presentation-type
                                  x-spacing y-spacing
                                  n-columns n-rows
                                  max-width max-height
                                  cell-align-x cell-align-y
                                  initial-spacing (row-wise \cl{t})
                                  (move-cursor \cl{t}) record-type}

This is a function interface to the item list formatter.  The elements of the
sequence \arg{items} are formatted as separate cells within the item list.

\arg{stream} is an output recording stream to which output will be done.  It
defaults to \cl{*standard-output*}.

\arg{printer} must be a function that takes two arguments, an item and a stream,
and outputs the item on the stream.  \arg{printer} has dynamic extent.  The
default for \arg{printer} is \cl{prin1}.

\arg{presentation-type} is a presentation-type.  When \arg{printer} is not
supplied, the items will be printed as if \arg{printer} were
\begin{verbatim}
#'(lambda (item stream)
    (present item presentation-type :stream stream))
\end{verbatim}
When \arg{printer} is supplied, each item will be enclosed in a presentation
whose type is \arg{presentation-type}.

\arg{x-spacing}, \arg{y-spacing}, \arg{n-columns}, \arg{n-rows},
\arg{max-width}, \arg{max-height}, \arg{initial-spacing}, \arg{row-wise}, and
\arg{move-cursor} are as for \cl{formatting-item-list}.

\arg{cell-align-x} and \arg{cell-align-y} are used to supply \cl{:align-x} and
\cl{:align-y} to an implicitly used \cl{formatting-cell}.

\arg{record-type} is as for \cl{formatting-item-list}.


\section {The Table and Item List Formatting Protocols}

Both table and item list formatting is implemented on top of the basic output
recording protocol, using \cl{with-new-output-record} to specify the appropriate
type of output record.  For example, \cl{formatting-table} first collects all
the output that belongs in the table into a collection of row, column, and cell
output records, all of which are children of a single table output record.
During this phase, \cl{stream-drawing-p} is bound to \cl{nil} and
\cl{stream-recording-p} is bound to \cl{t}.  When all the output has been
generated, the table layout constraint solver (\cl{adjust-table-cells} or
\cl{adjust-item-list-cells}) is called to compute the table layout, taking into
account such factors as the widest cell in a given column.  If the table is to
be split into multiple columns, \cl{adjust-multiple-columns} is now called.
Finally, the table output record is positioned on the stream at the current text
cursor position and then displayed by calling \cl{replay} on the table (or item
list) output record.


\subsection {Table Formatting Protocol}

Any output record class that implements the following generic functions is said
to support the table formatting protocol.

In the following subsections, the term ``non-table output records'' will be used
to mean any output record that is not a table, row, column, cell, or item list
output record.  When CLIM ``skips over intervening non-table output records'',
this means that it will bypass all the output records between two such table
output records (such as a table and a row, or a row and a cell) that are not
records of those classes (most notably, presentation output records).  CLIM
implementations are encouraged to detect invalid nesting of table output
records, such as a row within a row, a cell within a cell, or a row within a
cell.  Note that this does not prohibit the nesting of calls to
\cl{formatting-table}, it simply requires that programmers include the inner
table within one of the cells of the outer table.


\Defprotoclass {table-output-record}

The protocol class that represents tabular output records; a subclass of
\cl{output-record}.
\IfYouWantClass {a} {table output record} {table-output-record}

\Defpredicate {table-output-record-p} {object}

Returns \term{true} if \arg{object} is a \term{table output record}, otherwise
returns \term{false}.

\definitarg {:x-spacing}
\definitarg {:y-spacing}
\definitarg {:multiple-columns-x-spacing}
\Definitarg {:equalize-column-widths}

All subclasses of \cl{table-output-record} must handle these initargs, which are
used to specify, respectively, the $x$ and $y$ spacing, the multiple column $x$
spacing, and equal-width columns attributes of the table.

\Defclass {standard-table-output-record}

The instantiable class of output record that represents tabular output.  Its
children will be a sequence of either rows or columns, with presentation output
records possibly intervening.  This is a subclass of \cl{table-output-record}.

\Defgeneric {map-over-table-elements} {function table-record type}

Applies \arg{function} to all the rows or columns of \arg{table-record} that
are of type \arg{type}.  \arg{type} is either \cl{:row}, \cl{:column}, or
\cl{:row-or-column}.  \arg{function} is a function of one argument, an output
record; it has dynamic extent.  \cl{map-over-table-elements} is responsible
for ensuring that rows, columns, and cells are properly nested.  It must skip
over intervening non-table output record structure, such as presentations.

\Defgeneric {adjust-table-cells} {table-record stream}

This function is called after the tabular output has been collected, but before
it has been replayed.  The method on \cl{standard-table-output-record}
implements the usual table layout constraint solver (described above) by moving
the rows or columns of the table output record \arg{table-record} and the cells
within the rows or columns.    \arg{stream} is the stream on which the table is
displayed.

\Defgeneric {adjust-multiple-columns} {table-record stream}

This is called after \cl{adjust-table-cells} to account for the case where the
programmer wants the table to have multiple columns.  \arg{table-record} and
\arg{stream} are as for \cl{adjust-table-cells}.


\subsection {Row and Column Formatting Protocol}

Any output record class that implements the following generic functions is said
to support the row (or column) formatting protocol.

\Defprotoclass {row-output-record}

The protocol class that represents one row in a table; a subclass of
\cl{output-record}.
\IfYouWantClass {a} {row output record} {row-output-record}

\Defpredicate {row-output-record-p} {object}

Returns \term{true} if \arg{object} is a \term{row output record}, otherwise
returns \term{false}.

\Defclass {standard-row-output-record}

The instantiable class of output record that represents a row of output within a
table.  Its children will be a sequence of cells, and its parent (skipping
intervening non-tabular records such as presentations) will be a table output
record.  This is a subclass of \cl{row-output-record}.

\Defgeneric {map-over-row-cells} {function row-record}

Applies \arg{function} to all the cells in the row \arg{row-record}, skipping
intervening non-table output record structure.  \arg{function} is a function of
one argument, an output record corresponding to a table cell within the row; it
has dynamic extent.

\issue {PM} {All these map-* functions should mind the hierarchy, I guess. So if a table is put within a table cell ...}

\Defprotoclass {column-output-record}

The protocol class that represents one column in a table; a subclass of
\cl{output-record}.
\IfYouWantClass {a} {column output record} {column-output-record}

\Defpredicate {column-output-record-p} {object}

Returns \term{true} if \arg{object} is a \term{column output record}, otherwise
returns \term{false}.

\Defclass {standard-column-output-record}

The instantiable class of output record that represents a column of output within
a table.  Its children will be a sequence of cells, and its parent (skipping
intervening non-tabular records such as presentations) will be a table output
record; presentation output records may intervene.  This is a subclass of
\cl{column-output-record}.

\Defgeneric {map-over-row-cells} {function column-record}

Applies \arg{function} to all the cells in the column \arg{column-record},
skipping intervening non-table output record structure.  \arg{function} is a
function of one argument, an output record corresponding to a table cell within
the column; it has dynamic extent.


\subsection {Cell Formatting Protocol}

Any output record class that implements the following generic functions is said
to support the cell formatting protocol.

\Defprotoclass {cell-output-record}

The protocol class that represents one cell in a table or an item list; a
subclass of \cl{output-record}.
\IfYouWantClass {a} {cell output record} {cell-output-record}

\Defpredicate {cell-output-record-p} {object}

Returns \term{true} if \arg{object} is a \term{cell output record}, otherwise
returns \term{false}.

\definitarg {:align-x}
\definitarg {:align-y}
\definitarg {:min-width}
\Definitarg {:min-height}

All subclasses of \cl{cell-output-record} must handle these initargs, which are
used to specify, respectively, the $x$ and $y$ alignment, and the minimum width
and height attributes of the cell.

\Defclass {standard-cell-output-record}

The instantiable class of output record that represent a single piece of output
within a table row or column, or an item list.  Its children will either be
presentations, or output records that represent displayed output.  This is a
subclass of \cl{cell-output-record}.

\defgeneric {cell-align-x} {cell}
\defgeneric {cell-align-y} {cell}
\defgeneric {cell-min-width}  {cell}
\Defgeneric {cell-min-height} {cell}

These functions return, respectively, the $x$ and $y$ alignment and minimum
width and height of the \term{cell output record} \arg{cell}.


\subsection {Item List Formatting Protocol}

\Defprotoclass {item-list-output-record}

The protocol class that represents an item list; a subclass of
\cl{output-record}.
\IfYouWantClass {an} {item list output record} {item-list-output-record}

\Defpredicate {item-list-output-record-p} {object}

Returns \term{true} if \arg{object} is an \term{item list output record},
otherwise returns \term{false}.

\definitarg {:x-spacing}
\definitarg {:y-spacing}
\definitarg {:initial-spacing}
\definitarg {:row-wise}
\definitarg {:n-rows}
\definitarg {:n-columns}
\definitarg {:max-width}
\Definitarg {:max-height}

All subclasses of \cl{item-list-output-record} must handle these initargs, which
are used to specify, respectively, the $x$ and $y$ spacing, the initial spacing,
row-wise, the desired number of rows and columns, and maximum width and height
attributes of the item list.

\Defclass {standard-item-list-output-record}

The instantiable output record that represents item list output.  Its children
will be a sequence of cells, with presentations possibly intervening.  This is a
subclass of \cl{item-list-output-record}.

\Defgeneric {map-over-item-list-cells} {function item-list-record}

Applies \arg{function} to all of the cells in \arg{item-list-record}.
\cl{map-over-item-list-cells} must skip over intervening non-list output record
structure, such as presentations.  \arg{function} is a function of one argument,
an output record corresponding to a cell in the item list; it has dynamic
extent.

\Defgeneric {adjust-item-list-cells} {item-list-record stream}

This function is called after the item list output has been collected, but
before the record has been replayed.  The method on
\cl{standard-item-list-output-record} implements the usual item list layout
constraint solver.  \arg{item-list-record} is the item list output record, and
\arg{stream} is the stream on which the item list is displayed.

